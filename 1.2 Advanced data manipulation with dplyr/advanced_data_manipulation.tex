%!TEX TS-program = xelatex

\documentclass[t]{beamer}

\usetheme{Hannover}
\usecolortheme{rose}

%%% Работа с русским языком
\usepackage[english,russian]{babel}   %% загружает пакет многоязыковой вёрстки
\usepackage{fontspec,xltxtra,xunicode}      %% подготавливает загрузку шрифтов Open Type, True Type и др.
%\defaultfontfeatures{Ligatures={TeX},Renderer=Basic}  %% свойства шрифтов по умолчанию
\setmainfont[Ligatures={TeX,Historic},
SmallCapsFont={Brill},
SmallCapsFeatures={Letters=SmallCaps}]{Brill} %% задаёт основной шрифт документа
\setsansfont{Brill}                    %% задаёт шрифт без засечек
\setmonofont[Ligatures=NoCommon]{DejaVu Sans}
\newfontfamily\SYM{Brill}
\usepackage{indentfirst}
%%% Дополнительная работа с математикой
\usepackage{amsmath,amsfonts,amssymb,amsthm,mathtools} % AMS
\usepackage{icomma} % "Умная" запятая: $0,2$ --- число, $0, 2$ --- перечисление

%%% Работа с картинками
\usepackage{wrapfig} % Обтекание рисунков текстом
\usepackage{rotating}
\usepackage{fixltx2e}
\usepackage{hhline}
\usepackage{lscape}

%%% Работа с таблицами
\usepackage{array,tabularx,tabulary,booktabs} % Дополнительная работа с таблицами
\usepackage{longtable}  % Длинные таблицы
\usepackage{multirow} % Слияние строк в таблице

\usepackage{multicol} % Несколько колонок

%%% Страница
%\usepackage{fancyhdr} % Колонтитулы
% 	\pagestyle{fancy}
 	%\renewcommand{\headrulewidth}{0pt}  % Толщина линейки, отчеркивающей верхний колонтитул
% 	\lfoot{Нижний левый}
% 	\rfoot{Нижний правый}
% 	\rhead{Верхний правый}
% 	\chead{Верхний в центре}
% 	\lhead{Верхний левый}
%	\cfoot{Нижний в центре} % По умолчанию здесь номер страницы

\usepackage{setspace} % Интерлиньяж
%\onehalfspacing % Интерлиньяж 1.5
%\doublespacing % Интерлиньяж 2
\singlespacing % Интерлиньяж 1

\usepackage{subfig} % подкартинки
\usepackage{lastpage} % Узнать, сколько всего страниц в документе.
\usepackage{soul} % Модификаторы начертания
\usepackage{bbding}
\usepackage{hyperref}
\usepackage{tikz} % Работа с графикой
\usepackage{pgfplots}
\usepackage{pgfplotstable}
\usepackage{verbatim}

\usepackage{attachfile2}
 \attachfilesetup{appearance=true,
color=0 0 0
 }
\usepackage{alltt}

%%% Лингвистические пакеты
%\usepackage{savetrees} % пакет, который экономит место
\usepackage{forest} % для рисования деревьев
\usepackage{vowel} % для рисования трапеций гласных
\usepackage{natbib}
\bibpunct[: ]{[}{]}{;}{a}{}{,}
\usepackage[nogroupskip,nopostdot, nonumberlist]{glossaries}
%\usepackage{glossary-mcols} 
%\setglossarystyle{mcolindex}
\usepackage{philex} % пакет для примеров
\newcommand{\mytem}{\item[$\circ$]}
\addto\captionsrussian{
\renewcommand{\refname}{}}

\newcommand{\apostrophe}{\XeTeXglyph\XeTeXcharglyph"0027\relax}
\usetikzlibrary{patterns}

\usepackage{ulem}
\setbeamercolor{alerted text}{fg=blue}
\setbeamersize{text margin left=4mm,text margin right=1mm} 
\setbeamertemplate{navigation symbols}{
	\usebeamerfont{footline}%
    \usebeamercolor[fg]{footline}%
    \hspace{1em}%
    {{\small презентация доступна: \href{http://1drv.ms/1PMoiZj}{\textbf{http://1drv.ms/1PMoiZj}}}
    \hspace{40mm}
    \insertframenumber/\inserttotalframenumber\vspace{0.5mm}}}
% начало
\title[]{Продвинутая манипуляция с датафреймами:\\ пакеты \small \texttt{dplyr}\normalsize\ и \small\texttt{tidyr}\normalsize}
\author[]{Г. Мороз}
\date{}
\begin{document}
\frame{\titlepage}
\begin{frame}{Данные}
В данной презентации все примеры будут приводиться на примере датасета из работы [Chi-kuk 2007] (доступна по ссылке http://goo.gl/MKfSc6). В работе исследовались речь 7 гомосексуальных и 7 гетеросексуальных носителей кантонского диалекта языка юэ. В датасете есть следующие переменные:
\begin{itemize}
\mytem долгота s (s.duration.ms)
\mytem долгота гласных (vowel.duration.ms)
\mytem среднее значении ЧОТ (average.f0.Hz)
\mytem диапозон ЧОТ (f0.range.Hz)
\mytem сколько носителей воспринимает говорящего как гомосексуалa (perceived.as.homo)
\mytem сколько носителей воспринимает говорящего как гетеросексуалa (perceived.as.hetero)
\mytem ориентация говорящего (orientation)
\mytem возраст говорящего (age)
\end{itemize}
\end{frame}
\section{tbl\_df}
\begin{frame}{tbl\_df}
Первая важная функция пакет \scriptsize\verb"dplyr"\normalsize\ это формат tbl\_df, который позволяет выводить на экран значительно более удобным способом, а в остальном это все тот же датафрейм:\\
\vfill
\scriptsize
\begin{alltt}
homo <- read.csv("http://goo.gl/Zjr9aF") \hfill \# скачиваем данные \medskip\\
\alert{library(dplyr)}
homo <- \alert{tbl\_df(}homo\alert{)} \hfill \# преобразуем данные \\
head(homo) \hfill \# первые шесть значений \\
\vfill
\# A tibble: 6 x 10\\
\begin{tabular}{rrrrr}
 & speaker & s.duration.ms & vowel.duration.ms & average.f0.Hz \\ 
 & <fctr> & <dbl> & <dbl> & <dbl> \\ 
1 & A & 61.40 & 112.60 & 119.51  \\ 
2 & B & 63.90 & 126.49 & 100.29  \\ 
3 & C & 55.08 & 126.81 & 114.90 \\ 
4 & D & 78.11 & 119.17 & 126.61 \\ 
5 & E & 64.71 & 93.68 & 130.76  \\ 
6 & F & 67.00 & 127.87 & 150.79 \\ 
\end{tabular}
\\
\# ... with 6 more variables: f0.range.Hz <dbl>, \\
\#   perceived.as.homo <int>, perceived.as.hetero <int>, \\
\#   perceived.as.homo.percent <dbl>, orientation <fctr>, \\
\#   age <int>
\end{alltt}
\normalsize
\end{frame}
\section{подмножество строк}
\begin{frame}{Подмножество строк по условию}
\noindent Какие информанты старше 28?\\ \vfill
\begin{columns}[T] 
\begin{column}{.48\textwidth}
\textbf{base R}\\
\scriptsize
\begin{alltt}
homo[homo\$age > 28, ]
\end{alltt}
\normalsize
\end{column}
\hfill
\begin{column}{.48\textwidth}
\textbf{dplyr}\\
\scriptsize
\begin{alltt}
homo \alert{\%>\%}\\ 
\ \ \ \alert{filter(}age > 28\alert{)}
\end{alltt}
\normalsize
\end{column}
\end{columns}
\hfill
\scriptsize
\begin{alltt}
\# A tibble: 7 x 10\\
\begin{tabular}{rrrrr}
 & speaker & s.duration.ms & vowel.duration.ms & average.f0.Hz \\ 
 & <fctr> & <dbl> & <dbl> & <dbl> \\ 
1 & A & 61.40 & 112.60 & 119.51  \\ 
3 & C & 55.08 & 126.81 & 114.90 \\ 
4 & D & 78.11 & 119.17 & 126.61 \\ 
6 & F & 67.00 & 127.87 & 150.79 \\ 
5 & J & 59.59  &  121.01  & 123.90  \\ 
6 & K & 62.94 & 137.37 & 119.48 \\ 
7 & N & 57.67 & 118.02 & 121.48 \\ 
\end{tabular}
\\
\# ... with 6 more variables: f0.range.Hz <dbl>,\\
\#   perceived.as.homo <int>, perceived.as.hetero <int>,\\
\#   perceived.as.homo.percent <dbl>, orientation <fctr>,\\
\#   age <int>
\end{alltt}
\normalsize
\hfill\\
\noindent Оператор \scriptsize\texttt{\alert{\%>\%}}\normalsize\ называется конвейер (по-английски pipe) и вставляется при помощи сочетания клавиш \alert{Ctrl+Shift+m}.\bigskip
\end{frame}

\begin{frame}{Подмножество строк по номеру}
\noindent Какие информанты хранятся под номером 3, 4, 5, 6, 7?\\ \vfill
\begin{columns}[T] 
\begin{column}{.48\textwidth}
\textbf{base R}\\
\scriptsize
\begin{alltt}
homo[3:7, ]
\end{alltt}
\normalsize
\end{column}
\hfill
\begin{column}{.48\textwidth}
\textbf{dplyr}\\
\scriptsize
\begin{alltt}
homo \alert{\%>\%}\\ 
\ \ \ \alert{slice(}3:7\alert{)}
\end{alltt}
\normalsize
\end{column}
\end{columns}
\hfill
\scriptsize
\begin{alltt}
\# A tibble: 5 x 10\\
\begin{tabular}{rrrrr}
 & speaker & s.duration.ms & vowel.duration.ms & average.f0.Hz \\ 
 & <fctr> & <dbl> & <dbl> & <dbl> \\ 
1 & C & 55.08 & 126.81 & 114.90 \\ 
2 & D & 78.11 & 119.17 & 126.61 \\ 
3 & E & 64.71 & 93.68 & 130.76  \\ 
4 & F & 67.00 & 127.87 & 150.79 \\ 
5 & N & 57.67 & 118.02 & 121.48 \\ 
\end{tabular}
\\
\# ... with 6 more variables: f0.range.Hz <dbl>,\\
\#   perceived.as.homo <int>, perceived.as.hetero <int>,\\
\#   perceived.as.homo.percent <dbl>, orientation <fctr>,\\
\#   age <int>
\end{alltt}
\normalsize
\hfill\\
\end{frame}

\section{подмножество столбцов}
\begin{frame}{Подмножество столбцов по названию}
\noindent Выделяет столбцы с информантов, ориентацией и возрастом.\\ \vfill
\begin{columns}[T] 
\begin{column}{.65\textwidth}
\textbf{base R}\\
\scriptsize
\begin{alltt}
cbind.data.frame(homo\$speaker, homo\$age)
\end{alltt}
\normalsize
\end{column}
\hfill
\begin{column}{.29\textwidth}
\textbf{dplyr}\\
\scriptsize
\begin{alltt}
homo \alert{\%>\%}\\ 
\ \ \ \alert{select(}speaker, age\alert{)}
\end{alltt}
\normalsize
\end{column}
\end{columns}
\hfill
\scriptsize
\begin{alltt}
\# A tibble: 14 x 2 \\
\begin{tabular}{rrr}
 & speaker & age \\ 
 & <fctr> & <int> \\ 
1 & A & 30 \\ 
2 & B & 19 \\ 
3 & C & 29 \\ 
4 & D & 36 \\ 
5 & E & 27 \\ 
6 & F & 33 \\ 
7 & G & 28 \\ 
8 & H & 22 \\ 
9 & I & 22 \\ 
10 & J & 40 \\ 
11 & K & 30 \\ 
12 & L & 25 \\ 
13 & M & 20 \\ 
14 & N & 29 \\ 
\end{tabular}
\\
\end{alltt}
\normalsize
\end{frame}

\begin{frame}{Подмножество столбцов по номеру}
\noindent Как выбрать столбцы под номером 8, 9, 10?\\ \vfill
\begin{columns}[T] 
\begin{column}{.48\textwidth}
\textbf{base R}\\
\scriptsize
\begin{alltt}
homo[, 8:10]
\end{alltt}
\normalsize
\end{column}
\hfill
\begin{column}{.48\textwidth}
\textbf{dplyr}\\
\scriptsize
\begin{alltt}
homo \alert{\%>\%}\\ 
\ \ \ \alert{select(}8:10\alert{)}
\end{alltt}
\normalsize
\end{column}
\end{columns}
\hfill
\scriptsize
\begin{alltt}
\# A tibble: 14 x 3\\
\begin{tabular}{rrrr}
 & perceived.as.homo.percent & orientation & age \\ 
 & <dbl> & <fctr> & <int> \\ 
1 & 0.28 & hetero & 30 \\ 
2 & 0.80 & hetero & 19 \\ 
3 & 0.36 & homo & 29 \\ 
4 & 0.60 & homo & 36 \\ 
5 & 0.40 & homo & 27 \\ 
6 & 0.68 & homo & 33 \\ 
7 & 0.80 & hetero & 28 \\ 
8 & 0.84 & hetero & 22 \\ 
9 & 0.80 & homo & 22 \\ 
10 & 0.32 & homo & 40 \\ 
11 & 0.84 & homo & 30 \\ 
12 & 0.32 & hetero & 25 \\ 
13 & 0.36 & hetero & 20 \\ 
14 & 0.16 & hetero & 29 \\ 
\end{tabular}
\\
\end{alltt}
\normalsize
\hfill\\
\end{frame}

\begin{frame}{Подмножество столбцов по названию}
\noindent Выделяет столбцы от одного до другого.\\ \vfill
\textbf{dplyr}\\
\scriptsize
\begin{alltt}
homo \alert{\%>\%}\\ 
\ \ \ \alert{select(speaker:average.f0.Hz)} \hfill \# можно комбинировать оба способа\\
\end{alltt}
\normalsize
\hfill
\scriptsize
\begin{alltt}
\# A tibble: 14 x 4 \\
\begin{tabular}{rrrrr}
 & speaker & s.duration.ms & vowel.duration.ms & average.f0.Hz \\ 
 & <fctr> & <dbl> & <dbl> & <dbl> \\ 
1 & A & 61.40 & 112.60 & 119.51 \\ 
2 & B & 63.90 & 126.49 & 100.29 \\ 
3 & C & 55.08 & 126.81 & 114.90 \\ 
4 & D & 78.11 & 119.17 & 126.61 \\ 
5 & E & 64.71 & 93.68 & 130.76 \\ 
6 & F & 67.00 & 127.87 & 150.79 \\ 
7 & G & 65.39 & 147.52 & 128.96 \\ 
8 & H & 62.46 & 120.13 & 105.26 \\ 
9 & I & 60.45 & 140.44 & 109.86 \\ 
10 & J & 59.59 & 121.01 & 123.90 \\ 
11 & K & 62.94 & 137.37 & 119.48 \\ 
12 & L & 53.31 & 112.05 & 146.20 \\ 
13 & M & 45.13 & 133.74 & 155.34 \\ 
14 & N & 57.67 & 118.02 & 121.48 \\ 
\end{tabular}
\\
\end{alltt}
\normalsize
\end{frame}
\section{упорядочить строки}
\begin{frame}{Упорядочить строки по значению столбцов}
\noindent Упорядочивает строки сначала по ориентации, потом по возрасту.\\ \vfill
\begin{columns}[T] 
\begin{column}{.62\textwidth}
\textbf{base R}\\
\scriptsize
\begin{alltt}
homo[order(homo\$orientation, homo\$age), ]
\end{alltt}
\normalsize
\end{column}
\hfill
\begin{column}{.37\textwidth}
\textbf{dplyr}\\
\scriptsize
\begin{alltt}
homo \alert{\%>\%}\\ 
\ \ \ \alert{arrange(}orientation, age\alert{)}
\end{alltt}
\normalsize
\end{column}
\end{columns}
\hfill
\scriptsize
\begin{alltt}
\# A tibble: 14 x 3 \hfill \# выбраны только важные столбцы\\
\begin{tabular}{rrrr}
 & speaker & orientation & age \\ 
 & <fctr> & <fctr> & <int> \\ 
1 & B & hetero & 19 \\ 
2 & M & hetero & 20 \\ 
3 & H & hetero & 22 \\ 
4 & L & hetero & 25 \\ 
5 & G & hetero & 28 \\ 
6 & N & hetero & 29 \\ 
7 & A & hetero & 30 \\ 
8 & I & homo & 22 \\ 
9 & E & homo & 27 \\ 
10 & C & homo & 29 \\ 
11 & K & homo & 30 \\ 
12 & F & homo & 33 \\ 
13 & D & homo & 36 \\ 
14 & J & homo & 40 \\ 
\end{tabular}
\\
\end{alltt}
\normalsize
\hfill\\
\end{frame}

\begin{frame}{Упорядочить строки по значению столбцов}
\noindent В обратном порядке.\\ \vfill
\begin{columns}[T] 
\begin{column}{.60\textwidth}
\textbf{base R}\\
\scriptsize
\begin{alltt}
homo[order(\alert{\textbf{-}}homo\$age), ]
\end{alltt}
\normalsize
\end{column}
\hfill
\begin{column}{.37\textwidth}
\textbf{dplyr}\\
\scriptsize
\begin{alltt}
homo \alert{\%>\%}\\ 
\ \ \ \alert{arrange(}\alert{desc(}age\alert{))}
\end{alltt}
\normalsize
\end{column}
\end{columns}
\hfill
\scriptsize
\begin{alltt}
\# A tibble: 14 x 2 \hfill \# выбран только важный столбец\\
\begin{tabular}{rrr}
 & speaker & age \\ 
 & <fctr> & <int> \\ 
1 & J & 40 \\ 
2 & D & 36 \\ 
3 & F & 33 \\ 
4 & A & 30 \\ 
5 & K & 30 \\ 
6 & N & 29 \\ 
7 & C & 29 \\ 
8 & G & 28 \\ 
9 & E & 27 \\ 
10 & L & 25 \\ 
11 & H & 22 \\ 
12 & I & 22 \\ 
13 & M & 20 \\ 
14 & B & 19 \\ 
\end{tabular}
\\
\end{alltt}
\normalsize
\hfill\\
\end{frame}
\section{уникальные комбинации}
\begin{frame}{Уникальные комбинации строк}
\noindent Уникальные значения строк в столбце ориентация:
\begin{columns}[T] 
\begin{column}{.62\textwidth}
\textbf{base R}\\
\scriptsize
\begin{alltt}
unique(homo\$orientation) \bigskip \\
$\left[1\right]$ hetero homo \\
Levels: hetero homo
\end{alltt}
\normalsize
\end{column}
\hfill
\begin{column}{.37\textwidth}
\textbf{dplyr}\\
\scriptsize
\begin{alltt}
homo \alert{\%>\%}\\ 
\ \ \ \alert{distinct(}orientation\alert{)} \bigskip\\
\begin{tabular}{rrrr}
& orientation\\
1 & hetero\\
2 & homo\\
\end{tabular}
\\
\end{alltt}
\normalsize
\end{column}
\end{columns}
\end{frame}
\begin{frame}{Уникальные комбинации строк}
\noindent В случае, если хочется выбрать несколько столбцов, средства R становятся сложнее: \bigskip\\
\textbf{base R}
\scriptsize
\begin{alltt}
unique(homo[c("orientation", "perceived.as.homo")])\bigskip\\
\end{alltt}
\normalsize
\textbf{dplyr}
\scriptsize
\begin{alltt}
homo \alert{\%>\% }\\
\ \ \ \alert{distinct(}orientation, perceived.as.homo\alert{)}
\end{alltt}
\normalsize
\end{frame}
\section{добавить столбец}
\begin{frame}{Добавить и преобразовать столбцы}
Добавяет столбцы c минимумом и максимом F0 каждого носителя:\bigskip\\
\textbf{base R}
\scriptsize
\begin{alltt}
homo\$f0.min <- homo\$average.f0.Hz – homo\$f0.range.Hz/2\\
homo\$f0.max <- homo\$average.f0.Hz + homo\$f0.range.Hz/2
\bigskip\\
\end{alltt}
\normalsize
\textbf{dplyr}
\scriptsize
\begin{alltt}
homo \alert{\%>\% }\\
\ \ \ \alert{mutate(}\\
\ \ \ \ \ \ f0.min = average.f0.Hz – f0.range.Hz/2, \\
\ \ \ \ \ \ f0.max = average.f0.Hz + f0.range.Hz/2\alert{)}
\end{alltt}
\normalsize
\end{frame}
\begin{frame}{Добавить и преобразовать столбцы}
\scriptsize
\begin{alltt}
\# A tibble: 14 x 3 \hfill \# выбраны только важные столбцы\\
\begin{tabular}{rrrr}
 & speaker & f0.min & f0.max \\ 
 & <fctr> & <dbl> & <dbl> \\ 
1 & A & 93.26 & 145.76 \\ 
2 & B & 43.29 & 157.29 \\ 
3 & C & 63.30 & 166.50 \\ 
4 & D & 97.21 & 156.01 \\ 
5 & E & 112.06 & 149.46 \\ 
6 & F & 129.79 & 171.79 \\ 
7 & G & 69.86 & 188.06 \\ 
8 & H & 77.41 & 133.11 \\ 
9 & I & 61.66 & 158.06 \\ 
10 & J & 68.05 & 179.75 \\ 
11 & K & 75.68 & 163.28 \\ 
12 & L & 117.30 & 175.10 \\ 
13 & M & 105.09 & 205.59 \\ 
14 & N & 102.78 & 140.18 \\ 
\end{tabular}
\end{alltt}
\normalsize
\end{frame}
\section{tidyr}
\begin{frame}{tidyr}
В основе пакета \scriptsize\verb"tidyr"\normalsize\ лежит понятие \textbf{Tidy Data}:
\begin{itemize}
\mytem каждая переменная — колонка
\mytem каждое наблюдение — строчка
\mytem все данные, связанные с одними тем же типом измерения, собраны в одну таблицу
\end{itemize}
Именно такой формат \textit{ожидают} большинство статистических функций и функций машинного обучения.
\end{frame}
\begin{frame}{tidyr}
\textbf{df.short}\\
\scriptsize
\begin{alltt}
\begin{tabular}{rrrrr}
 & consonant & initial & intervocalic & final \\ 
1 & stops & 123 & 57 & 30 \\ 
2 & fricatives & 87 & 77 & 69 \\ 
3 & affricates & 73 & 82 & 12 \\ 
4 & nasals & 7 & 78 & 104 \\ 
\end{tabular}
\end{alltt}
\normalsize
\textbf{df.long}\\
\scriptsize
\begin{alltt}
\begin{tabular}{rrrr}
 & consonant & position & number \\ 
1 & stops & initial & 123 \\ 
2 & fricatives & initial & 87 \\ 
3 & affricates & initial & 73 \\ 
4 & nasals & initial & 7 \\ 
5 & stops & intervocalic & 57 \\ 
6 & fricatives & intervocalic & 77 \\ 
7 & affricates & intervocalic & 82 \\ 
8 & nasals & intervocalic & 78 \\ 
9 & stops & final & 30 \\ 
10 & fricatives & final & 69 \\ 
11 & affricates & final & 12 \\ 
12 & nasals & final & 104 \\ 
\end{tabular}
\end{alltt}
\normalsize
\end{frame}
\begin{frame}[fragile]{Short format $\Leftrightarrow$ Long format:\\gather() and spread()}
\scriptsize
\begin{alltt}
df.short <- data.frame(
                            consonant = c("stops", "fricatives", "affricates", "nasals"),
                            initial = c(123, 87, 73, 7),
                            intervocal = c(57, 77, 82, 78),
                            final = c(30, 69, 12, 104)) \bigskip
\alert{library(tidyr)}
df.long <- \hfill # short to long
      df.short %>%
      \alert{gather}(position, number, initial:final) \bigskip
df.short<- \hfill # long to short
      df.long %>%
      \alert{spread}(position, number)
\end{alltt}
\normalsize
\end{frame}
\section{summarise и group\_by}
\begin{frame}{Пакет dplyr: summarise и group\_by}
Функция \scriptsize\verb"summarise()"\normalsize\ (или \scriptsize\texttt{summari\textbf{\alert{z}}e()}\normalsize) позволяет получить любые описательные статистики, которые доступны в R, в целом не отличается от них практически ничем.
\scriptsize
\begin{alltt}
homo \%>\%\\
\ \ \ \alert{summarise}(min(age), mean(s.duration.ms)) \bigskip\\
\# A tibble: 1 x 2\\
\begin{tabular}{rrr}
&  min(age) & mean(s.duration.ms)\\
&     <int>             &  <dbl>\\
1  &     19  &          61.22429 \\
\end{tabular}
\end{alltt}
\normalsize
\end{frame}
\begin{frame}{Пакет dplyr: summarise и group\_by}
В сочетании с командой \scriptsize\verb"group\_by()"\normalsize, которая группирует данные по какому-то параметру/параметрам, данная функция становится мощным инструментом.
\begin{columns}[T] 
\begin{column}{.38\textwidth}
\scriptsize
\begin{alltt}
homo \%>\%\\
\ \ \ \alert{group\_by}(orientation) \\
\ \ \ \alert{summarise}(count = n())\\
\# A tibble: 2 x 2\\
\begin{tabular}{rrr}
&  orientation & count\\
&     <fctr>             &  <int>\\
1  &     hetero  &         7\\
2  &     homo &          7\\
\end{tabular}
\end{alltt}
\end{column}
\hfill
\begin{column}{.58\textwidth}
\scriptsize
\begin{alltt}
homo \%>\%\\
\ \ \ \alert{group\_by}(orientation) \\
\ \ \ \alert{summarise}(mean(s.duration.ms))\\
\# A tibble: 2 x 2\\
\begin{tabular}{rrr}
&  orientation & mean(s.duration.ms)\\
&     <fctr>             &  <dbl>\\
1  &     hetero  &         58.46571\\
2  &     homo &          63.98286\\
\end{tabular}
\end{alltt}
\normalsize
\end{column}
\end{columns}
\vfill
В \textbf{base R} подобное можно сделать при помощи функции \scriptsize\verb"aggregate()"\normalsize:
\scriptsize
\begin{alltt}
\alert{aggregate}(speaker\textasciitilde orientation, length, data = homo)
\end{alltt}
\normalsize
\end{frame}
\section{}
\begin{frame}
{\huge Спасибо за внимание!\bigskip\\
\normalsize Пишите письма\\
agricolamz@gmail.com
\vspace{-130pt}}
\end{frame}
\end{document}