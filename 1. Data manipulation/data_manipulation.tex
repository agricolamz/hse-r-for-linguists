%!TEX TS-program = xelatex

\documentclass[t]{beamer}

\usetheme{Hannover}
\usecolortheme{rose}

%%% Работа с русским языком
\usepackage[english,russian]{babel}   %% загружает пакет многоязыковой вёрстки
\usepackage{fontspec,xltxtra,xunicode}      %% подготавливает загрузку шрифтов Open Type, True Type и др.
%\defaultfontfeatures{Ligatures={TeX},Renderer=Basic}  %% свойства шрифтов по умолчанию
\setmainfont[Ligatures={TeX,Historic},
SmallCapsFont={Brill},
SmallCapsFeatures={Letters=SmallCaps}]{Brill} %% задаёт основной шрифт документа
\setsansfont{Brill}                    %% задаёт шрифт без засечек
\setmonofont[Ligatures=NoCommon]{DejaVu Sans}
\newfontfamily\SYM{Brill}
\usepackage{indentfirst}
%%% Дополнительная работа с математикой
\usepackage{amsmath,amsfonts,amssymb,amsthm,mathtools} % AMS
\usepackage{icomma} % "Умная" запятая: $0,2$ --- число, $0, 2$ --- перечисление

%%% Работа с картинками
\usepackage{wrapfig} % Обтекание рисунков текстом
\usepackage{rotating}
\usepackage{fixltx2e}
\usepackage{hhline}
\usepackage{lscape}

%%% Работа с таблицами
\usepackage{array,tabularx,tabulary,booktabs} % Дополнительная работа с таблицами
\usepackage{longtable}  % Длинные таблицы
\usepackage{multirow} % Слияние строк в таблице

\usepackage{multicol} % Несколько колонок

%%% Страница
%\usepackage{fancyhdr} % Колонтитулы
% 	\pagestyle{fancy}
 	%\renewcommand{\headrulewidth}{0pt}  % Толщина линейки, отчеркивающей верхний колонтитул
% 	\lfoot{Нижний левый}
% 	\rfoot{Нижний правый}
% 	\rhead{Верхний правый}
% 	\chead{Верхний в центре}
% 	\lhead{Верхний левый}
%	\cfoot{Нижний в центре} % По умолчанию здесь номер страницы

\usepackage{setspace} % Интерлиньяж
%\onehalfspacing % Интерлиньяж 1.5
%\doublespacing % Интерлиньяж 2
\singlespacing % Интерлиньяж 1

\usepackage{subfig} % подкартинки
\usepackage{lastpage} % Узнать, сколько всего страниц в документе.
\usepackage{soul} % Модификаторы начертания
\usepackage{bbding}
\usepackage{hyperref}
\usepackage{tikz} % Работа с графикой
\usepackage{pgfplots}
\usepackage{pgfplotstable}
\usepackage{verbatim}

\usepackage{attachfile2}
 \attachfilesetup{appearance=true,
color=0 0 0
 }
\usepackage{alltt}

%%% Лингвистические пакеты
%\usepackage{savetrees} % пакет, который экономит место
\usepackage{forest} % для рисования деревьев
\usepackage{vowel} % для рисования трапеций гласных
\usepackage{natbib}
\bibpunct[: ]{[}{]}{;}{a}{}{,}
\usepackage[nogroupskip,nopostdot, nonumberlist]{glossaries}
%\usepackage{glossary-mcols} 
%\setglossarystyle{mcolindex}
\usepackage{philex} % пакет для примеров
\newcommand{\mytem}{\item[$\circ$]}
\addto\captionsrussian{
\renewcommand{\refname}{}}

\newcommand{\apostrophe}{\XeTeXglyph\XeTeXcharglyph"0027\relax}
\usetikzlibrary{patterns}
\usepackage{mathtools}
\usepackage{ulem}
\setbeamersize{text margin left=4mm,text margin right=1mm} 
\setbeamertemplate{navigation symbols}{
	\usebeamerfont{footline}%
    \usebeamercolor[fg]{footline}%
    \hspace{1em}%
    {{\small презентация доступна: \href{http://1drv.ms/1TDPR6m}{\textbf{http://1drv.ms/1TDPR6m}}}
    \hspace{38mm}
    \insertframenumber/\inserttotalframenumber\vspace{0.5mm}}}
% начало
\title[]{Типы объектов в R и манипуляция с ними}
\author[]{Г. Мороз}
\date{}
\begin{document}
\frame{\titlepage}
\section{об R}
\begin{frame}{Что  такое R?}
\begin{itemize}
\mytem среда для статистического анализа, обработки и визуализации данных
\item[\texttt{\symbol{"1F63C}}] язык программирования \pause (хотя многие скептичны\dots)
\end{itemize}
\begin{itemize}
\mytem \href{http://www.r-project.org/}{{\color{red!13!blue}{скачать R}}}
\mytem \href{http://www.rstudio.com/}{{\color{red!13!blue}{скачать RStudio}}}
\mytem для любителей командной строки есть \href{http://dirk.eddelbuettel.com/code/littler.html}{{\color{red!13!blue}{\verb"littler"}}}
\mytem \href{http://sourceforge.net/projects/npptor/}{{\color{red!13!blue}{R in Notepad ++}}}, \href{http://manuals.bioinformatics.ucr.edu/home/programming-in-r/vim-r}{{\color{red!13!blue}{Vim-R-Tmux}}}, \href{https://rkward.kde.org/}{{\color{red!13!blue}{RKWard}}}\dots
\end{itemize}
\pause Не обязательно в R:\\
\begin{itemize}
\mytem \href{http://www.mathworks.com/products/matlab/}{\includegraphics[height=2ex]{matlab.jpg}~Matlab}
\mytem \href{https://www.gnu.org/software/octave/}{\includegraphics[height=2.7ex]{octave.png}~Octave}
\mytem \href{http://www.statsoft.ru/}{\includegraphics[height=2.3ex]{statistica.jpg}~STATISTICA} (\href{http://www.statosphere.ru/home.html}{{\color{red!13!blue}{и русский блог об этой программе}}})
\mytem \href{http://www-01.ibm.com/software/analytics/spss/}{\includegraphics[height=2ex]{spss.jpg}~SPSS}
\mytem \href{http://juliastats.github.io/}{\includegraphics[height=2.7ex]{julia.png}~Julia}
\mytem и др.
\end{itemize}
\end{frame}
\begin{frame}{Преимущества R}
\begin{itemize}
\mytem бесплатный
\mytem с открытым исходным кодом
\mytem идет под многими операционными системами
\mytem с кучей пакетов (< 6500), к которым написан хэлп с примерами
\mytem реализована вся статистика, визуализация и обработка данных
\mytem понимает уйму форматов
\mytem пишет во множество форматов
\mytem можно проиллюстрировать все свои манипуляции с данными в R Markdown
\mytem большое комьюнити $\Rightarrow$ быстро ответят на вопрос, на многое уже отвечено, много книжек
\end{itemize}
\end{frame}
\begin{frame}{Недостатки}
\begin{itemize}
\mytem руководство не всегда на высшем уровне
\mytem во многих функциях не уделяется большое внимание расходу памяти
\mytem работа с некоторыми типами данных хуже (например с символами)
\end{itemize}
\vfill
Посмотрите статью \href{http://blog.dominodatalab.com/comparing-python-and-r-for-data-science/}{{\color{red!13!blue}{Comparing Python and R for Data Science}}}
\end{frame}
\section{калькулятор}
\begin{frame}[fragile]{Калькулятор}
\footnotesize
\verb"(4+1)*7+9-4" \hfill \# $(4+1)\times 7+9-4$ \\ \vfill
\verb"57 %% 43" \hfill \# остаток от деления \\ \vfill
\verb"(35+7)/-Inf" \hfill \# {\large $\frac{35+7}{-\infty}$} \\ \vfill
\verb"2^2^3" или \verb"2**2**3" \hfill \# $2^{2^3}$ \\ \vfill
\verb"abs(-5)" \hfill \# $|-5|$ \\ \vfill
\verb"log(100, 10)" \hfill \# log$_{10}(100)$ \\ \vfill
\verb"log2(64)", \verb"log10(100)" \hfill \# log$_2(64)$, log$_{10}(100)$\\ \vfill
\verb"sum(1:6, 7, 10:8)" \hfill \# $\sum\limits_{n=1}^{10}  n$\\ \vfill
\verb"prod(1:6, 7, 10:8)" \hfill \# $\prod\limits_{n=1}^{10}  n$\\ \vfill
\verb"factorial(8)" \hfill \# $8!$\\ \vfill
\verb"choose(7, 3)" \hfill \# {\large $3 \choose 7$}\\
\end{frame}
\section{объекты}
\begin{frame}{Атомарные и базовые объекты}
\begin{itemize}
\mytem атомарные объекты
\begin{itemize}
\mytem numeric — число \hfill по умолчанию
\mytem integer — целое число \hfill  5L
\mytem complex  — комплексное число \hfill 7+2i
\mytem character  — символ
\mytem logical (TRUE, FALSE) — логический оператор
\mytem NA (not available) — пропущенное значение
\mytem NULL
\mytem NaN (not a number)
\end{itemize}
\mytem базовые объекты
\begin{itemize}
\mytem vector — вектор \hfill{\color{red!13!blue}{объекты только одного класса}}
\mytem matrix — вектор с двумя измерениями (матрица)
\mytem array — вектор с двумя и более измерений (многомер. матрица)
\mytem list — список \hfill{\color{red!13!blue}{может содержать объекты разного класса}}
\mytem dataframe — датафрейм
\end{itemize}
\mytem комментирование
\begin{itemize}
\mytem \footnotesize\verb"\#"\normalsize
\mytem В RStudio \beamerbutton{Ctrl} + \beamerbutton{Shift} + \beamerbutton{C} или \beamerbutton{Command} + \beamerbutton{Shift} + \beamerbutton{C}
\end{itemize}
\end{itemize}
\end{frame}
\subsection{векторы}
\begin{frame}[fragile]{Векторы: базовое}
\footnotesize 
\begin{alltt}
a <- 3 \hfill #\beamerbutton{Alt} + \beamerbutton{-}\medskip
a \hfill # смотрим значение переменной\medskip
a + 1\medskip
a \hfill  # значение не изменилось\medskip
a + 3 -> a\medskip
a\medskip
b <- "ку-ку"\medskip
c <- 1:6\medskip
d <- c(1:3, 2, 2, 3)\medskip
e <- c(100:1)*2\medskip
f <- pi\medskip
g <- c(Маша = 1, Ваня = 2, Саша = 2, Аня = 2) \hfill # вектор с именами
\end{alltt}
\end{frame}
\begin{frame}[fragile]{Векторы: последовательности, генераторы чисел}
Фрагменты арифметических последовательностей можно создавать при помощи функции \footnotesize\verb"seq()"\normalsize \\
\footnotesize 
\begin{alltt}
3:23 \hfill {\color{red!13!blue}{seq(3, 23)}}
(3:23)*3\hfill seq(9, 69, {\color{red!13!blue}{by = 3}})\hfill seq(9, 69, {\color{red!13!blue}{3}})\hfill \# от 9 до 69 по 3
seq(3, 2, {\color{red!13!blue}{length=100}}) \hfill \# сто элементов от 3 до 2
\end{alltt}
\normalsize
\vfill
Можно создавать векторы из повторяющихся элементов:
\footnotesize 
\begin{alltt}
rep(5, 2) \hfill \# 2 раза 5
rep(33:22, 4) \hfill \# 4 раза 33:22
rep(33:22, {\color{red!13!blue}{each = 4}}) \hfill \# 4 раза каждый элемент 33:22
\end{alltt}
\normalsize
\vfill
Можно включить генератор случайных чисел:
\footnotesize 
\begin{alltt}
runif(1, 11, 37) \hfill \# случайное число в промежутке (11, 37)
runif(5, 11, 37) \hfill \# пять случайных чисел в промежутке (11, 37)
sample(11:37, 6) \hfill \# шесть случайных целых чисел из 11:37
sample(11:37, 6, {\color{red!13!blue}{replace=TRUE}}) \hfill \# можно повторяться
sample({\color{red!13!blue}{c("a", "u", "o", "y")}}, 6, replace=TRUE) 
\end{alltt}
\normalsize
\end{frame}
\subsection{матрицы}
\begin{frame}[fragile]{Матрицы}
\begin{itemize}
\mytem A <- matrix(c(2, 4, 3, 1, 5, 7), \hfill \# данные \\
{\color{red!13!blue}{  ncol=3) \hfill \# количество колонок}}
\footnotesize 
\begin{alltt}
     [,1] [,2] [,3]
[1,]    2    3    5
[2,]    4    1    7
\end{alltt}
\normalsize
\vfill
\mytem B <- matrix(c(2, 4, 3, 1, 5, 7), \hfill \# данные \\
{\color{red!13!blue}{  nrow=2)              \hfill \# количество строк}}
\footnotesize 
\begin{alltt}
     [,1] [,2] [,3]
[1,]    2    3    5
[2,]    4    1    7
\end{alltt}
\normalsize
\vfill
\mytem C <- matrix(c(2, 4, 3, 1, 5, 7), \hfill \# данные \\
  nrow=2,              \hfill \# количество строк\\
  {\color{red!13!blue}{  byrow = TRUE) \hfill \# по строчкам}}
\footnotesize 
\begin{alltt}
     [,1] [,2] [,3]
[1,]    2    4    3
[2,]    1    5    7
\end{alltt}
\normalsize
\end{itemize}
\end{frame}
\begin{frame}[fragile]{Матрицы}
\begin{multicols}{2}
\begin{itemize}
\mytem A*10 \medskip
\footnotesize 
\begin{alltt}
     [,1] [,2] [,3]
[1,]   20   30   50
[2,]   40   10   70
\end{alltt}
\normalsize
\bigskip
\mytem транспозиция матрицы \\ t(C) \medskip
\footnotesize 
\begin{alltt}
     [,1] [,2]
[1,]    2    1
[2,]    4    5
[3,]    3    7
\end{alltt}
\normalsize
\bigskip
\mytem A+C \medskip
\footnotesize 
\begin{alltt}
     [,1] [,2] [,3]
[1,]    4    7    8
[2,]    5    6   14
\end{alltt}
\normalsize
\columnbreak
\mytem поэлементное умножение \\ A*C  \medskip
\footnotesize 
\begin{alltt}
     [,1] [,2] [,3]
[1,]    4   12   15
[2,]    4    5   49
\end{alltt}
\normalsize
\vfill
\mytem умножение матриц \\ A \%*\% t(C) \medskip
\footnotesize 
\begin{alltt}
     [,1] [,2]
[1,]   31   52
[2,]   33   58
\end{alltt}
\normalsize
\vfill
\mytem декомпозиция матрицы \\ c(A) \medskip
\footnotesize 
\begin{alltt}
[1] 2 4 3 1 5 7
\end{alltt}
\normalsize
\end{itemize}
\end{multicols}
\end{frame}

\begin{frame}[fragile]{Матрицы: размерности}
У любой матрицы есть размерности и их сообщает функция \footnotesize\verb"dim()"\normalsize:\\
dim(A)
\footnotesize 
\begin{alltt}
[1] 2 3
\end{alltt}
\normalsize
dim(t(B))
\footnotesize 
\begin{alltt}
[1] 3 2
\end{alltt}
\vfill
\normalsize
m <- c(2, 4, 3, 1, 5, 7) \hfill \# создадим вектор\\
dim(m) <- c(2, 3) \hfill \# изменим его размерности
\footnotesize 
\begin{alltt}
     [,1] [,2] [,3]
[1,]    2    3    5
[2,]    4    1    7
\end{alltt}
\normalsize
\end{frame}
\subsection{многомерные матрицы}
\begin{frame}[fragile]{Многомерные матрицы}
array в R — это матрица с любым количеством размерностей (нет, это не массив):\\
A <- array(26:3, dim = c(3, 4, 2))
\footnotesize 
\begin{alltt}
, , 1
     [,1] [,2] [,3] [,4]
[1,]   26   23   20   17
[2,]   25   22   19   16
[3,]   24   21   18   15

, , 2
     [,1] [,2] [,3] [,4]
[1,]   14   11    8    5
[2,]   13   10    7    4
[3,]   12    9    6    3
\end{alltt}
\normalsize
c(A) \hfill \# декомпозиция
\footnotesize 
\begin{alltt}
[1] 26 25 24 23 22 21 20 19 18 17 16 15 14 13 12 11 10  9  8  7  6  5  4  3
\end{alltt}
\normalsize
\end{frame}
\subsection{списки}
\begin{frame}[fragile]{Списки}
Векторы, матрицы и многомерные матрицы позволяют иметь лишь объекты одного класса. Чтобы совмещать единицы разного класса используется функция \footnotesize\verb"list()"\normalsize.\\
list("abc"{}, TRUE, NA, c(666, 333))
\footnotesize 
\begin{alltt}
[[1]]
[1] "abc" \medskip
[[2]]
[1] TRUE \medskip
[[3]]
[1] NA \medskip
[[4]]
[1] 666 333
\end{alltt}
\normalsize
\begin{itemize}
\mytem объектами в \footnotesize\verb"list()"\normalsize{} трудно манипулировать
\mytem объекты в \footnotesize\verb"list()"\normalsize{} упорядочены
\mytem объекты, содержащие объект \footnotesize\verb"list()"\normalsize{}, тоже относятся к типу \footnotesize\verb"list()"\normalsize
\end{itemize}
\end{frame}
\subsection{датафрейм}
\begin{frame}[fragile]{Датафреймы}
\textit{Датафрейм} (иногда называют \textit{таблицей данных}) — одномерный список векторов одинаковой длинны.\\
data.frame(num = c(2,3,8,1),
           tf = c(T,F,T,T),
           let = c("a","f","k","z"))
\footnotesize 
\begin{alltt}
  num        tf let
1     2  TRUE   a
2     3 FALSE   f
3     8  TRUE   k
4     1  TRUE   z
\end{alltt}
\normalsize
\end{frame}
\subsection{факторы}
\begin{frame}[fragile]{Факторы}
В R существует отдельный класс векторов, для кодирования единиц из номинальных (род существительного) или порядковых шкал (оценки грамматичности предложения). Иначе векторы c("m"{}, "n"{}, "f") или c(5, 4, 3) воспринимаются R как строки или числа, что может в некоторых случаях вызвать ошибку. Для преобразования вектора в фактор используется следующая функция:\bigskip\\
{\color{red!13!blue}{factor(}}c("f"{}, "f"{}, "m"{}, "n"{}, "n"{}, "f"{}, "m"{}, "f"), {\color{red!13!blue}{levels}} = c("m"{}, "n"{}, "f"))
\footnotesize
\begin{alltt}
[1] f f m n n f m f
Levels: m n f
\end{alltt}
\normalsize
\end{frame}
\subsection{пакеты}
\begin{frame}{Пакеты}
Большинство дополнительных функции в R содержаться в дополнительных пакетах.
\begin{itemize}
\mytem install.packages(“{\color{red!13!blue}{pkg}}”) \hfill \# скачиваем пакет
\mytem update.packages \hfill \# проверяет обновления
\mytem library(“{\color{red!13!blue}{pkg}}”) \hfill \# добавляет пакет в сессию
\mytem detach(ʺpackage:{\color{red!13!blue}{pkg}}ʺ) \hfill \# удаляет пакет из сессии
\end{itemize}
\end{frame}
\section{if, else, for, while}
\begin{frame}{Сравнение и логические операторы}
\footnotesize \verb"m < n" \normalsize \hfill \# меньше \\ \vfill
\footnotesize \verb"m > n" \normalsize \hfill \# больше\\ \vfill
\footnotesize \verb"m <= n" \normalsize \hfill \# меньше или равно\\ \vfill
\footnotesize \verb"m >= n"\normalsize  \hfill \# больше или равно\\ \vfill
\footnotesize \verb"m == n" \normalsize \hfill \# равно\\ \vfill
\footnotesize \verb"m != n" \normalsize \hfill \# не равно\\ \vfill
\footnotesize \verb"m \& n" \normalsize \hfill \# и\\ \vfill
\footnotesize \verb"m | n" \normalsize \hfill \# или\\ \vfill
\footnotesize \verb"xor(m, n)" \normalsize \hfill \# исключительное или\\
\end{frame}
\begin{frame}[fragile]{Таблицы истинности}
a <- c (TRUE, TRUE, FALSE)\\
b <- c (TRUE, FALSE, FALSE)\\
outer(a, b, "\&")\hfill \# и\\
\footnotesize 
\begin{alltt}
      [,1]  [,2]  [,3]
[1,]  TRUE FALSE FALSE
[2,]  TRUE FALSE FALSE
[3,] FALSE FALSE FALSE
\end{alltt}
\normalsize
outer(a, b, "\verb,|,")\hfill \# или\\
\footnotesize 
\begin{alltt}
     [,1]  [,2]  [,3]
[1,] TRUE  TRUE  TRUE
[2,] TRUE  TRUE  TRUE
[3,] TRUE FALSE FALSE
\end{alltt}
\normalsize
outer(a, b, "xor")\hfill \# исключительное или\\
\footnotesize 
\begin{alltt}
      [,1]  [,2]  [,3]
[1,] FALSE  TRUE  TRUE
[2,] FALSE  TRUE  TRUE
[3,]  TRUE FALSE FALSE
\end{alltt}
\normalsize
\end{frame}
\begin{frame}[fragile]{Логические операции}
h <- 1:10\\
h
\footnotesize 
\begin{alltt}
[1]  {\color{red!13!blue}{1  2  3  4}}  5  6  7  {\color{red!27!green}{8  9 10}}
\end{alltt}
\normalsize
h > 4
\footnotesize 
\begin{alltt}
[1] {\color{red!13!blue}{FALSE FALSE FALSE FALSE}}   TRUE  TRUE  TRUE {\color{red!27!green}{TRUE TRUE TRUE}}
\end{alltt}
\normalsize
h < 8
\footnotesize 
\begin{alltt}
[1] {\color{red!13!blue}{TRUE TRUE TRUE TRUE}}   TRUE  TRUE  TRUE {\color{red!27!green}{FALSE FALSE FALSE}}
\end{alltt}
\normalsize
h > 4 \& h < 8 \hfill \# и\\
\footnotesize 
\begin{alltt}
[1] {\color{red!13!blue}{FALSE FALSE FALSE FALSE}}  TRUE  TRUE  TRUE {\color{red!27!green}{FALSE FALSE FALSE}}
\end{alltt}
\normalsize
h > 4 | h < 8 \hfill \# или\\
\footnotesize 
\begin{alltt}
[1] {\color{red!13!blue}{TRUE TRUE TRUE TRUE}} TRUE TRUE TRUE {\color{red!27!green}{TRUE TRUE TRUE}}
\end{alltt}
\normalsize
xor(h > 4, h < 8) \hfill \# исключительное или\\
\footnotesize 
\begin{alltt}
[1]  {\color{red!13!blue}{TRUE  TRUE  TRUE  TRUE}} FALSE FALSE FALSE  {\color{red!27!green}{TRUE  TRUE  TRUE}}
\end{alltt}
\normalsize
\end{frame}
\begin{frame}[fragile]{Циклы}
\begin{itemize}
\mytem {\color{red!13!blue}{if}}(condition)\{cmd1\} {\color{red!27!green}{else}}\{cmd2\}
\footnotesize 
\begin{alltt}
{\color{red!13!blue}{if}}(i %% 2 == 0)\{
  print("even")
  \} {\color{red!27!green}{else}} \{print("odd")\}
\end{alltt}
\normalsize
\vfill
\mytem {\color{red!13!blue}{ifelse}}(condition, cmd1, cmd2)
\footnotesize 
\begin{alltt}
{\color{red!13!blue}{ifelse}}(i %% 2 == 0, "even", "odd")
\end{alltt}
\normalsize
\vfill
\mytem {\color{red!13!blue}{for}}(variable {\color{red!27!green}{in}} x)\{cmd\}
\footnotesize 
\begin{alltt}
{\color{red!13!blue}{for}}(i {\color{red!27!green}{in}} 10:20)\{
  print(exp(i))\}
\end{alltt}
\normalsize
\vfill
\mytem {\color{red!13!blue}{while}}(condition)\{cmd\}
\footnotesize 
\begin{alltt}
{\color{red!13!blue}{while}}(i < 7)\{
  print(exp(i))
  i <- i+1\}
\end{alltt}
\normalsize
\end{itemize}
\vfill
R считается медленным языком, так что осторожнее с циклами. \href{http://adv-r.had.co.nz/Performance.html}{\color{red!13!blue}{В книге H. Wickham Advanced R}}{} говорится, почему так, и как бороться.
\end{frame}
\begin{frame}[fragile]{Пустые переменные}
При работе с циклами часто бывает полезно заранее создать пустую переменную, в которую будут писаться значения. \vfill
\begin{itemize}
\mytem векторы
\begin{itemize}
\mytem numeric(p) \hfill \# создает вектор типа numeric длины p
\mytem integer(p) \hfill \# создает вектор типа integer длины p
\mytem complex(p) \hfill \# создает вектор типа complex длины p
\mytem character(p) \hfill \# создает вектор типа character длины p
\mytem logical(p) \hfill \# создает вектор типа logical длины p
\mytem rep(NA\_real\_, p) \hfill \# создает вектор типа integer длины p
\mytem rep(NA\_integer\_, p) \hfill \# создает вектор типа integer длины p
\mytem rep(NA\_complex\_, p) \hfill \# создает вектор типа complex длины p
\mytem rep(NA\_character\_, p) \hfill \# создает вектор типа character длины p
\end{itemize} \vfill
\mytem vector(mode = "list", length = 10) \hfill \# 10-элементный список \vfill
\mytem array(NA, dim = c(p, q, \dots)) \hfill \# матрицы c измерениями p, q, \dots \vfill
\mytem датафреймы
\end{itemize}
\end{frame}
\section{манипуляции}
\subsection{чтение}
\begin{frame}[fragile]{Чтение}
\begin{itemize}
\mytem команды \footnotesize \verb"read.table" \normalsize и \footnotesize \verb"read.csv" \normalsize почти идентичны
\footnotesize
\begin{alltt}
X <- {\color{red!13!blue}{read.table}}("/agricolamz/work/R/ALex.txt",  \hfill # путь
    sep = "\textbackslash{}t", \hfill # тип разделителя
    header = TRUE) \hfill # первая строчка — имена столбцов?\bigskip
X <- {\color{red!13!blue}{read.csv}}("http://goo.gl/GkfYYt", \hfill # url
    sep = "\textbackslash{}t", \hfill # тип разделителя
    header = TRUE) \hfill # первая строчка — имена столбцов?\bigskip
\end{alltt}
\normalsize
\mytem можно при помощи функции \footnotesize \verb"setwd()" \normalsize установить рабочую директорию и обращаться к файлам по имени
\footnotesize
\begin{alltt}
{\color{red!13!blue}{setwd}}("/agricolamz/work/R") \hfill # путь
X <- read.table("ALex.txt", \hfill  # имя файла
    sep = "\textbackslash{}t", \hfill # тип разделителя
    header = TRUE) \hfill # первая строчка — имена столбцов?\bigskip
\end{alltt}
\normalsize
\mytem для текстов есть команда \footnotesize \verb"readLines()" \normalsize
\footnotesize
\begin{alltt}
text <- {\color{red!13!blue}{readLines}}("test-2.txt", \hfill  # имя файла
    n = 10) \hfill # количество абзацев
\end{alltt}
\normalsize
\end{itemize}
\end{frame}
\subsection{разведка}
\begin{frame}{Разведка}
\begin{itemize}
\mytem \footnotesize \verb"?"\normalsize\hfill ставится перед функцией
\mytem \footnotesize \verb"apropos("" \verb"x""\verb")"\normalsize\hfill ищет, все, содержащее "x"
\mytem \footnotesize \verb"summary(x)"\normalsize\hfill для данных
\mytem \footnotesize \verb"str(x)"\normalsize\hfill и для данных, и для функций
\mytem \footnotesize \verb"head(x)"\normalsize\hfill первые 6 значений объекта
\mytem \footnotesize \verb"tail(x)"\normalsize\hfill последние 6 значений объекта
\mytem \footnotesize \verb"typeof(x)"\normalsize\hfill возвращает тип данных
\mytem \footnotesize \verb"is.что-то(x)"\normalsize\hfill возвращает вектор логических значений
\mytem \footnotesize \verb"table(x)"\normalsize\hfill возвращает таблицу сопряженности
\mytem \footnotesize \verb"unique(x)"\normalsize\hfill возвращает данные без дублей
\mytem \footnotesize \verb"max(x)"\normalsize, \footnotesize \verb"min(x)"\normalsize \hfill возвращает максим./миним. значение
\mytem \footnotesize\verb"length(x)"\normalsize, \footnotesize \verb"nrow(x)"\normalsize, \footnotesize \verb"ncol(x)"\normalsize\hfill количество элементов
\mytem \footnotesize \verb"ls()"\normalsize\hfill возвращает все переменные данной сессии
\mytem \footnotesize \verb"getAnywhere(x)"\normalsize\hfill возвращает код функции (не для всех)
\end{itemize}
\end{frame}
\subsection{indexing}
\begin{frame}{Indexing: векторы}
x <- c(43:21)
\begin{itemize}
\mytem x[3]\hfill \# в квадратных скобках\\
\footnotesize \verb"[1] 41"\normalsize \hfill
{\color{red!13!blue}{Любители Python! Индексация начинается с 1!}}
\mytem c(43:21)[3]\hfill \# аналогично можно с любым вектором\\
\footnotesize \verb"[1] 41"\normalsize
\mytem x[x\%\%7 == 0]\hfill \# логическое выражение\\
\footnotesize \verb"[1] 42 35 28 21"\normalsize
\mytem x[x\%\%7 == 0 \& x\%\%13==0]\hfill \# логическое выражение\\
\footnotesize \verb"[1] 42 28"\normalsize
\mytem x[-с(3:19)]\hfill \# все кроме элементов 3:19\\
\footnotesize \verb"[1] 43 42 24 23 22 21"\normalsize
\end{itemize}
\vfill
x <- c(Anna = 23, Wanda = 24, Agnieszka  = 28) \hfill \# вектор с именами
\begin{itemize}
\mytem x["Anna"]\hfill \# обращение по имени\\
\footnotesize \verb"Anna"\normalsize\\
\footnotesize \verb"23"\normalsize
\end{itemize}
\end{frame}
\begin{frame}[fragile]{Indexing: матрицы}
x <- matrix(9:60, nrow = 4)
\begin{itemize}
\mytem x[3] \hfill \# третий элемент\\
\footnotesize \verb"[1] 11"\normalsize
\mytem x[3,4]  \hfill \# третий столбец, четвертая строчка\\
\footnotesize \verb"[1] 23"\normalsize
\mytem x[3, ]  \hfill \# элементы третьей строчки\\
\footnotesize \verb"[1] 11 15 19 23 27 31 35 39 43 47 51 55 59"\normalsize
\mytem x[, 3] \hfill \# элементы третьего столбца\\
\footnotesize \verb"[1] 17 18 19 20"\normalsize
\mytem x[-c(1, 4),] \hfill \# все, кроме первой и четвертой строчек\\
\footnotesize
\begin{alltt}
     [,1] [,2] [,3] [,4] [,5] [,6] [,7] [,8] [,9] [,10] [,11] [,12] [,13]
[1,]   10   14   18   22   26   30   34   38   42    46    50    54    58
[2,]   11   15   19   23   27   31   35   39   43    47    51    55    59
\end{alltt}
\normalsize
\end{itemize}
\end{frame}
\begin{frame}[fragile]{Indexing: списки и датафреймы}
Объект \footnotesize\verb"list()"\normalsize{} содержит несколько векторов. Поэтому чтобы обратиться к какому-то из векторов объекта \footnotesize\verb"list()"\normalsize{} рекомендуется использовать двойные  квадратные скобки: \medskip\\
y <- list(tf = c(T, F), num = c(82, 99, 21), com = 2 + 9i)\\
y[[2]] \hfill \# второй вектор списка
\footnotesize
\begin{alltt}
[1] 82 99 21
\end{alltt}
\normalsize
Чтобы обратиться к элементу вектора следует использовать {\color{red!13!blue}{только двойные скобки}}, для обращения к векторы, и одинарные для обращения к элементу:\medskip\\
y{\color{red!13!blue}{[[2]][3]}} \hfill \# третий элемент второго вектора
\footnotesize
\begin{alltt}
[1] 21
\end{alltt}
\normalsize
Все это работает рекурсивно, так как объекты \footnotesize\verb"list()"\normalsize{}  могут быть вложены друг в друга. Это распространяется и на обращения по имени.
\end{frame}
\begin{frame}[fragile]{Indexing: датафреймы}
Обращение к элементам датафрейма устроены так же, как и в \footnotesize\verb"list()"\normalsize{} и \footnotesize\verb"matrix()"\normalsize{}:\\
z <- data.frame(tf = c(T, F), num = c(82, 99), com = c(2 + 9i, 8 + 3i))\\
z[[2]]
\footnotesize
\begin{alltt}
[1] 82 99
\end{alltt}
\normalsize
z[1, 2]
\footnotesize
\begin{alltt}
[1] 82
\end{alltt}
\normalsize
К векторам можно обращаться по имени: \\
z\$com
\footnotesize
\begin{alltt}
[1] 2+9i 8+3i
\end{alltt}
\normalsize
z\$com[2]
\footnotesize
\begin{alltt}
[1] 8+3i
\end{alltt}
\normalsize
Кроме того, важно отметить, что к элементам можно обращаться, указывая лишь часть имени, например, первую букву:\\
z\$c[2]
\footnotesize
\begin{alltt}
[1] 8+3i
\end{alltt}
\normalsize
\end{frame}
\subsection{соединение}
\begin{frame}[fragile]{Соединение}
\begin{itemize}
\mytem векторы: \footnotesize\verb"с(a, b)"\normalsize
\mytem матрицы: \footnotesize\verb"rbind(a, b), cbind(a, b)"\normalsize, многомерные — \footnotesize\verb"abind()"\normalsize
\mytem списки:
\end{itemize}
\vspace{-5mm}
\begin{multicols}{3}
{\color{red!13!blue}{a <- list(2, 3)}}\\
{\color{red!27!green}{b <- list(5, 6)}}\\
c({\color{red!13!blue}{a}}, {\color{red!27!green}{b}})
\footnotesize
\begin{alltt}
{\color{red!13!blue}{[[1]]
    [1] 2
[[2]]
    [1] 3}}
{\color{red!27!green}{[[3]]
    [1] 5
[[4]]
    [1] 6}}
\end{alltt}
\normalsize
\columnbreak
{\color{red!13!blue}{a <- list(2, 3)}}\\
{\color{red!27!green}{b <- list(5, 6)}}\\
c({\color{red!13!blue}{list(a)}},{\color{red!27!green}{list(b)}})
\footnotesize
\begin{alltt}
{\color{red!13!blue}{[[1]]
    [[1]][[1]]
        [1] 2
    [[1]][[2]]
        [1] 3}}
{\color{red!27!green}{[[2]]
    [[2]][[1]]
        [1] 5
    [[2]][[2]]
        [1] 6}}
\end{alltt}
\normalsize
\columnbreak
{\color{red!13!blue}{a <- list(2, 3)}}\\
{\color{red!27!green}{b <- list(5, 6)}}\\
mapply(c, {\color{red!13!blue}{a}}, {\color{red!27!green}{b}}, SIMPLIFY=F)
\footnotesize
\begin{alltt}
{\color{red!13!blue}{[[1]]
    [1] 2 5}}
{\color{red!27!green}{[[2]]
    [1] 3 6}}
\end{alltt}
\normalsize
\end{multicols}
\vspace{-3mm}
\begin{itemize}
\mytem датафреймы: \footnotesize\verb"rbind.data.frame(a, b), cbind.data.frame(a, b)"\normalsize
\end{itemize}
\end{frame}
\subsection{сортировки}
\begin{frame}[fragile]{Сортировки: вектора}
a <- c(1, 3, 1, 2, 2, 7, 4, 4, 2, 2, 2, 3, 9, 10, 2, 3, 4, 6, 5)\\
rev(a)
\footnotesize
\begin{alltt}
[1]  5  6  4  3  2 10  9  3  2  2  2  4  4  7  2  2  1  3  1
\end{alltt}
\normalsize
sort(a)
\footnotesize
\begin{alltt}
[1]  1  1  2  2  2  2  2  2  3  3  3  4  4  4  5  6  7  9 10
\end{alltt}
\normalsize
sort(a, decreasing = T)
\footnotesize
\begin{alltt}
[1] 10  9  7  6  5  4  4  4  3  3  3  2  2  2  2  2  2  1  1
\end{alltt}
\normalsize
sort(a, index.return = T)
\footnotesize
\begin{alltt}
\$x
[1]  1  1  2  2  2  2  2  2  3  3  3  4  4  4  5  6  7  9 10
\$ix
[1]  1  3  4  5  9 10 11 15  2 12 16  7  8 17 19 18  6 13 14
\end{alltt}
\normalsize
\end{frame}
\begin{frame}[fragile]{Сортировки: матрицы}
Так как матрицы — всего лишь вектора с измерениями, то все команды сортировки затрагивают их целиком и "убивают"{} измерения.\\
m <- matrix(c(4, 2, 3, 5, 1, 2, 6, 2, 6, 3, 1, 9), nrow = 4)\\
sort(m)
\footnotesize
\begin{alltt}
[1] 1 1 2 2 2 3 3 4 5 6 6 9
\end{alltt}
\normalsize
В связи с этим, если мы хотим сохранить структуру, измерения получившегося вектор следует изменить под старые:\\
p <- sort(m); dim(p) <- dim(m)\\
p
\footnotesize
\begin{alltt}
     [,1] [,2] [,3]
[1,]    1    2    5
[2,]    1    3    6
[3,]    2    3    6
[4,]    2    4    9
\end{alltt}
\normalsize
\end{frame}
\begin{frame}[fragile]{Сортировки: матрицы}
В пакет \footnotesize\verb"BioPhysConnectoR"\normalsize{} реализована функция \footnotesize\verb"mat.sort"\normalsize{} позволяющая сортировать матрицы по выбранной колонке:\\
library("BioPhysConnectoR")
\begin{multicols}{3}
b\\
\footnotesize
\begin{alltt}
     [,1] [,2] [,3]
[1,]    4    1    6
[2,]    2    2    3
[3,]    3    6    1
[4,]    5    2    9
\end{alltt}
\normalsize
mat.sort(b)\\
\footnotesize
\begin{alltt}
     [,1] [,2] [,3]
[1,]    2    2    3
[2,]    3    6    1
[3,]    4    1    6
[4,]    5    2    9
\end{alltt}
\normalsize
mat.sort(b, 3)\\
\footnotesize
\begin{alltt}
     [,1] [,2] [,3]
[1,]    3    6    1
[2,]    2    2    3
[3,]    4    1    6
[4,]    5    2    9
\end{alltt}
\normalsize
\end{multicols}
\end{frame}
\begin{frame}[fragile]{Сортировки: датафрейма}
Для сортировки матрицы используется команда \footnotesize\verb"order()"\normalsize\\
df <- data.frame(\\
name = c("Agnieszka"{}, "Marek"{}, "Marta"{}, "Magda"{}, "Jan"), \hfill \# имя\\
age = c(24, 23, 19, 25, 21), \hfill \# возраст\\
sex = c("f"{}, "m"{}, "f"{}, "f"{}, "m"), \hfill \# пол\\
lang = c(2, 2, 1, 4, 2)) \hfill \# владение языками
\begin{multicols}{2}
df[order(df[4]),]
df[order(df\$lang),]
\footnotesize
\begin{alltt}
           name age sex  lang
3        Marta    19     f        1
1 Agnieszka    24     f        2
2       Marek    23     m      2
5            Jan    21     m      2
4       Magda   25     f        4
\end{alltt}
\normalsize
df[order(-df[4], df[2]),]\\
df[order(-df\$lang, df\$age),]
\footnotesize
\begin{alltt}
           name age sex  lang
4       Magda   25     f        4
5            Jan    21     m      2
2       Marek    23     m      2
1 Agnieszka    24     f        2
3        Marta    19     f        1
\end{alltt}
\normalsize
\end{multicols}
\end{frame}
\subsection{reshaping}
\begin{frame}[fragile]{Преобразования}
df <- data.frame(\\
name = c("Agnieszka"{}, "Marek"{}, "Marta"{}, "Magda"{}, "Jan"), \hfill \# имя\\
age = c(24, 23, 19, 25, 21), \hfill \# возраст\\
sex = c("f"{}, "m"{}, "f"{}, "f"{}, "m"), \hfill \# пол\\
lang = c(2, 2, 1, 4, 2)) \hfill \# владение языками\\
\begin{multicols}{3}
table(df\$sex)\\
\footnotesize
\begin{alltt}
f m 
3 2 
\end{alltt}
\normalsize
table(df\$sex, df\$lang)
\footnotesize
\begin{alltt}
       4   1   2
  f    1   1   1
  m  0   0   2
\end{alltt}
\normalsize
prop.table(table(df\$s))\\
\footnotesize
\begin{alltt}
   f     m 
0.6  0.4 
\end{alltt}
\normalsize
\end{multicols}
as.data.frame(table(df\$sex, df\$lang))
\footnotesize
\begin{alltt}
  Var1 Var2 Freq
1    f       1       1
2    m     1       0
3    f       2       1
4    m     2       2
5    f       4       1
6    m     4       0
\end{alltt}
\normalsize
\end{frame}
\subsection{удаление NA}
\begin{frame}[fragile]{Удаление NA}
\begin{itemize}
\mytem вектор:\\
a <- c(1, 3, 5, NA, 1, 3, NA)\\
a[!is.na(a)] или a[complete.cases(a)]\\
\footnotesize
\begin{alltt}
[1] 1 3 5 1 3
\end{alltt}
\normalsize
\vfill
\mytem датафрейм:\\
z <- data.frame(tf = c(T, F, NA, T),\\
num = c(82, NA, 99, 56),\\
com = c(2 + 9i, 8 + 3i, 5 + 2i, 6 + 1i))\\
z[complete.cases(z),]\\
\footnotesize
\begin{alltt}
    tf num  com
1 TRUE  82 2+9i
4 TRUE  56 6+1i
\end{alltt}
\normalsize
\end{itemize}
\end{frame}
\subsection{запись}
\begin{frame}[fragile]{Запись файлов}
\begin{itemize}
\mytem write.csv(df, file = "/agricolamz/work/R/calculi.csv") \hfill \# путь \vfill
Можно при помощи функции setwd() установить рабочую директорию и не прописывать весь путь:
\mytem write.{\color{red!13!blue}{table}}(df, file = "calculi.csv"{}, {\color{red!13!blue}{sep = ";"{}}}) \hfill \# разделитель \vfill
\mytem для текстов есть команда \footnotesize\verb"writeLines"\normalsize
\end{itemize}
\end{frame}
\subsection{память и время}
\begin{frame}[fragile]{Расход времени и памяти}
\begin{itemize}
\mytem Sys.time()\\
\mytem system.time()\bigskip\\
library("pryr")
\mytem object{\_}size()
\mytem mem{\_}used()
\mytem mem{\_}change()
\end{itemize}
\end{frame}
\section{функции}
\begin{frame}{Функции}
Кроме встроенных функций пользователю также предоставлена возможность делать функции самостоятельно:\\
function(аргументы)\{действия\}\\
t <- function(x)\{factorial(x)\} \hfill \# функция с одной переменной\\
u <- function(x, y)\{x + y\} \hfill \# функция с двумя переменными\\
v <- function(x = 0)\{exp(x)\} \hfill \# функция со значением по умолчанию\\
\begin{multicols}{2}
v()\\
\footnotesize\verb"[1] 2.718282"\normalsize\\
v(4)\\
\footnotesize\verb"[1] 54.59815"\normalsize
\end{multicols}
\# умолчание одной переменной может зависеть от другой\\
w <- function(x=1, y = x*10)\{sum(x:y)\}\\
\begin{multicols}{3}
w()\\
\footnotesize\verb"[1] 55"\normalsize\\
w(4)\\
\footnotesize\verb"[1] 814"\normalsize\\
w(4, 2)\\
\footnotesize\verb"[1] 9"\normalsize\\
\end{multicols}
(function(x)\{3*x\})(10) \hfill \# анонимная функция
\end{frame}
\begin{frame}{Функции}
To understand computations in R, two slogans are helpful:
\begin{itemize}
\mytem Everything that exists is an object.
\mytem Everything that happens is a function call.
\end{itemize} 
\hfill — John Chambers\\ \vfill
Косые кавычки \verb"`"\dots\verb"`" позволяют обращаться к именам всех функций, даже с нестандартным синтаксисом.\\
x <- 10; y <- 5; x + y \hfill \# сложим два вектора\\
`+`(x, y) \hfill \# тот же результат\bigskip \\
for (i in 9:6) print(i) \hfill \# цикл\\
`for`(i, 1:2, print(i))\hfill \# тот же результат\bigskip \\
c(5:8)[3]\hfill \# третий элемент вектора\\
`[`(c(5:8), 3)\hfill \# тот же результат
\end{frame}
\section{}
\begin{frame}
{\huge Спасибо за внимание\bigskip\\
\normalsize Пишите письма\\
agricolamz@gmail.com
\vspace{-130pt}}
\end{frame}
\end{document}