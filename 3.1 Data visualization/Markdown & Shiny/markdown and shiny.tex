%!TEX TS-program = xelatex

\documentclass[t]{beamer}

\usetheme{Hannover}
\usecolortheme{rose}

%%% Работа с русским языком
\usepackage[english,russian]{babel}   %% загружает пакет многоязыковой вёрстки
\usepackage{fontspec,xltxtra,xunicode}      %% подготавливает загрузку шрифтов Open Type, True Type и др.
%\defaultfontfeatures{Ligatures={TeX},Renderer=Basic}  %% свойства шрифтов по умолчанию
\setmainfont[Ligatures={TeX,Historic},
SmallCapsFont={Brill},
SmallCapsFeatures={Letters=SmallCaps}]{Brill} %% задаёт основной шрифт документа
\setsansfont{Brill}                    %% задаёт шрифт без засечек
\setmonofont[Ligatures=NoCommon]{DejaVu Sans}
\newfontfamily\SYM{Brill}
\usepackage{indentfirst}
%%% Дополнительная работа с математикой
\usepackage{amsmath,amsfonts,amssymb,amsthm,mathtools} % AMS
\usepackage{icomma} % "Умная" запятая: $0,2$ --- число, $0, 2$ --- перечисление

%%% Работа с картинками
\usepackage{wrapfig} % Обтекание рисунков текстом
\usepackage{rotating}
\usepackage{fixltx2e}
\usepackage{hhline}
\usepackage{lscape}

%%% Работа с таблицами
\usepackage{array,tabularx,tabulary,booktabs} % Дополнительная работа с таблицами
\usepackage{longtable}  % Длинные таблицы
\usepackage{multirow} % Слияние строк в таблице

\usepackage{multicol} % Несколько колонок

%%% Страница
%\usepackage{fancyhdr} % Колонтитулы
% 	\pagestyle{fancy}
 	%\renewcommand{\headrulewidth}{0pt}  % Толщина линейки, отчеркивающей верхний колонтитул
% 	\lfoot{Нижний левый}
% 	\rfoot{Нижний правый}
% 	\rhead{Верхний правый}
% 	\chead{Верхний в центре}
% 	\lhead{Верхний левый}
%	\cfoot{Нижний в центре} % По умолчанию здесь номер страницы

\usepackage{setspace} % Интерлиньяж
%\onehalfspacing % Интерлиньяж 1.5
%\doublespacing % Интерлиньяж 2
\singlespacing % Интерлиньяж 1

\usepackage{subfig} % подкартинки
\usepackage{lastpage} % Узнать, сколько всего страниц в документе.
\usepackage{soul} % Модификаторы начертания
\usepackage{bbding}
\usepackage{hyperref}
\usepackage{tikz} % Работа с графикой
\usepackage{pgfplots}
\usepackage{pgfplotstable}
\usepackage{verbatim}

\usepackage{attachfile2}
 \attachfilesetup{appearance=true,
color=0 0 0
 }
\usepackage{alltt}

%%% Лингвистические пакеты
%\usepackage{savetrees} % пакет, который экономит место
\usepackage{forest} % для рисования деревьев
\usepackage{vowel} % для рисования трапеций гласных
\usepackage{natbib}
\bibpunct[: ]{[}{]}{;}{a}{}{,}
\usepackage[nogroupskip,nopostdot, nonumberlist]{glossaries}
%\usepackage{glossary-mcols} 
%\setglossarystyle{mcolindex}
\usepackage{philex} % пакет для примеров
\newcommand{\mytem}{\item[$\circ$]}
\addto\captionsrussian{
\renewcommand{\refname}{}}

\newcommand{\apostrophe}{\XeTeXglyph\XeTeXcharglyph"0027\relax}
\usetikzlibrary{patterns}

\usepackage{ulem}
\setbeamersize{text margin left=4mm,text margin right=1mm} 
\setbeamertemplate{navigation symbols}{
	\usebeamerfont{footline}%
    \usebeamercolor[fg]{footline}%
    \hspace{1em}%
    {{\small презентация доступна: \href{http://1drv.ms/1PMoiZj}{\textbf{http://1drv.ms/1PMoiZj}}}
    \hspace{40mm}
    \insertframenumber/\inserttotalframenumber\vspace{0.5mm}}}
% начало
\title[]{Markdown и Shiny}
\author[]{Г. Мороз}
\date{}
\begin{document}
\frame{\titlepage}
\section{R Markdown}
\begin{frame}[fragile]{R Markdown}
R Markdown позволяет создавать документы, содержащие отчет или презентацию с прямым вызовом функций R и его выдачей.
\begin{itemize}
\mytem Инсталяция: \scriptsize {\color{red!13!blue}{\verb'install.packages("rmarkdown")'}} \normalsize
\mytem жанры: документ, презентация
\mytem доступные форматы: html, pdf (если стоит TeX), docx
\mytem доступно легкое форматирование
\end{itemize}
\begin{multicols}{2}
\small
\begin{itemize}
\mytem \scriptsize {\color{red!13!blue}{\verb"*италик*"}} \normalsize
\mytem \scriptsize {\color{red!13!blue}{\verb"_италик_"}} \normalsize
\mytem \scriptsize {\color{red!13!blue}{\verb"**жирный**"}} \normalsize
\mytem \scriptsize {\color{red!13!blue}{\verb"__жирный__"}} \normalsize
\mytem \scriptsize {\color{red!13!blue}{\verb"^надписное^"}} \normalsize
\mytem \scriptsize {\color{red!13!blue}{\verb"~~перечеркнутое~~"}} \normalsize
\mytem \scriptsize {\color{red!13!blue}{\verb"# 1. Заголовок"}} \normalsize
\mytem \scriptsize {\color{red!13!blue}{\verb"## 1.2 Подзаголовок"}} \normalsize
\mytem \LaTeX : \scriptsize {\color{red!13!blue}{\verb"$\frac{4}{3}$"}} \normalsize
\mytem Горизонтальная линия: \scriptsize {\color{red!13!blue}{\verb"***"}} \normalsize
\mytem \scriptsize {\color{red!13!blue}{\verb"> отдельный блок"}}
\mytem \scriptsize {\color{red!13!blue}{\verb"* булет"}}
\mytem \scriptsize {\color{red!13!blue}{\verb"+ подбулет"}}
\end{itemize}
\normalsize
\end{multicols}
\end{frame}
\begin{frame}[fragile]{R Markdown: оформление кода}
\begin{multicols}{2}
\begin{itemize}
\mytem код и результат
\scriptsize 
\begin{alltt}
{\color{red!13!blue}{```\{r\}}}
plot(99:22)
{\color{red!13!blue}{```}}
\end{alltt}
\normalsize
\mytem только код
\scriptsize 
\begin{alltt}
{\color{red!13!blue}{```\{r eval = FALSE\}}}
plot(99:22)
{\color{red!13!blue}{```}}
\end{alltt}
\normalsize
\mytem только результат
\scriptsize 
\begin{alltt}
{\color{red!13!blue}{```\{r echo = FALSE\}}}
plot(99:22)
{\color{red!13!blue}{```}}
\end{alltt}
\normalsize
\columnbreak
\mytem в тексте
\tiny 
\begin{alltt}
\$5 \textbackslash choose 7\$ будет {\color{red!13!blue}{`r choose(7, 5)`}}
\end{alltt}
$5 \choose 7$ будет 21
\normalsize
\mytem без предупреждений
\scriptsize 
\begin{alltt}
{\color{red!13!blue}{```\{r warning = FALSE\}}}
plot(99:22)
{\color{red!13!blue}{```}}
\end{alltt}
\normalsize
\mytem без сообщений
\scriptsize 
\begin{alltt}
{\color{red!13!blue}{```\{r message = FALSE\}}}
plot(99:22)
{\color{red!13!blue}{```}}
\end{alltt}
\normalsize
\end{itemize}
\end{multicols}
\end{frame}
\section{Shiny}
\begin{frame}[fragile]{Shiny}
Shiny позволяет делать интерактивные приложения в R.
\end{frame}
\section{}
\begin{frame}
{\huge Спасибо за внимание!\bigskip\\
\normalsize Пишите письма\\
agricolamz@gmail.com
\vspace{-130pt}}
\end{frame}
\end{document}