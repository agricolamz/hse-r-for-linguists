%!TEX TS-program = xelatex

\documentclass[t]{beamer}

\usetheme{Hannover}
\usecolortheme{rose}

%%% Работа с русским языком
\usepackage[english,russian]{babel}   %% загружает пакет многоязыковой вёрстки
\usepackage{fontspec,xltxtra,xunicode}      %% подготавливает загрузку шрифтов Open Type, True Type и др.
%\defaultfontfeatures{Ligatures={TeX},Renderer=Basic}  %% свойства шрифтов по умолчанию
\setmainfont[Ligatures={TeX,Historic},
SmallCapsFont={Brill},
SmallCapsFeatures={Letters=SmallCaps}]{Brill} %% задаёт основной шрифт документа
\setsansfont{Brill}                    %% задаёт шрифт без засечек
\setmonofont[Ligatures=NoCommon]{DejaVu Sans}
\newfontfamily\SYM{Brill}
\usepackage{indentfirst}
%%% Дополнительная работа с математикой
\usepackage{amsmath,amsfonts,amssymb,amsthm,mathtools} % AMS
\usepackage{icomma} % "Умная" запятая: $0,2$ --- число, $0, 2$ --- перечисление

%%% Работа с картинками
\usepackage{wrapfig} % Обтекание рисунков текстом
\usepackage{rotating}
\usepackage{fixltx2e}
\usepackage{hhline}
\usepackage{lscape}

%%% Работа с таблицами
\usepackage{array,tabularx,tabulary,booktabs} % Дополнительная работа с таблицами
\usepackage{longtable}  % Длинные таблицы
\usepackage{multirow} % Слияние строк в таблице

\usepackage{multicol} % Несколько колонок

%%% Страница
%\usepackage{fancyhdr} % Колонтитулы
% 	\pagestyle{fancy}
 	%\renewcommand{\headrulewidth}{0pt}  % Толщина линейки, отчеркивающей верхний колонтитул
% 	\lfoot{Нижний левый}
% 	\rfoot{Нижний правый}
% 	\rhead{Верхний правый}
% 	\chead{Верхний в центре}
% 	\lhead{Верхний левый}
%	\cfoot{Нижний в центре} % По умолчанию здесь номер страницы

\usepackage{setspace} % Интерлиньяж
%\onehalfspacing % Интерлиньяж 1.5
%\doublespacing % Интерлиньяж 2
\singlespacing % Интерлиньяж 1

\usepackage{subfig} % подкартинки
\usepackage{lastpage} % Узнать, сколько всего страниц в документе.
\usepackage{soul} % Модификаторы начертания
\usepackage{bbding}
\usepackage{hyperref}
\usepackage{tikz} % Работа с графикой
\usepackage{pgfplots}
\usepackage{pgfplotstable}
\usepackage{verbatim}

\usepackage{attachfile2}
 \attachfilesetup{appearance=true,
color=0 0 0
 }
\usepackage{alltt}

%%% Лингвистические пакеты
%\usepackage{savetrees} % пакет, который экономит место
\usepackage{forest} % для рисования деревьев
\usepackage{vowel} % для рисования трапеций гласных
\usepackage{natbib}
\bibpunct[: ]{[}{]}{;}{a}{}{,}
\usepackage[nogroupskip,nopostdot, nonumberlist]{glossaries}
%\usepackage{glossary-mcols} 
%\setglossarystyle{mcolindex}
\usepackage{philex} % пакет для примеров
\newcommand{\mytem}{\item[$\circ$]}
\addto\captionsrussian{
\renewcommand{\refname}{}}

\newcommand{\apostrophe}{\XeTeXglyph\XeTeXcharglyph"0027\relax}
\usetikzlibrary{patterns}

\usepackage{ulem}
\setbeamercolor{alerted text}{fg=yellow}
\setbeamersize{text margin left=4mm,text margin right=1mm} 
\setbeamertemplate{navigation symbols}{
	\usebeamerfont{footline}%
    \usebeamercolor[fg]{footline}%
    \hspace{1em}%
    {{\small презентация доступна: \href{http://1drv.ms/1PMoiZj}{\textbf{http://1drv.ms/1PMoiZj}}}
    \hspace{40mm}
    \insertframenumber/\inserttotalframenumber\vspace{0.5mm}}}
% начало
\title[]{Работа со строками в R}
\author[]{Г. Мороз}
\date{}
\begin{document}
\frame{\titlepage}
\section{создание}
\begin{frame}[fragile]{Как получить строку?}
\begin{itemize}
\mytem a <- "the quick brown fox jumps over the lazy dog"
\mytem a <- 'the quick brown fox jumps over the lazy dog'
\mytem a <- "the quick {\color{red!13!blue}{'brown'{}}} fox jumps over the lazy dog"
\mytem a <- 'the quick {\color{red!13!blue}{"brown"{}}} fox jumps over the lazy dog'
\mytem b <- {}"{}"{}
\mytem c <- character(3) \hfill \# создает пустые строки\\
\footnotesize
\begin{alltt}
[1] "" "" ""
\end{alltt}
\normalsize
\mytem d <- as.character(3) \hfill \# превращает в строку\\
\footnotesize
\begin{alltt}
[1] "3"
\end{alltt}
\normalsize
\mytem d <- as.character(c(4:9)) \hfill \# вектор в строку\\
\footnotesize
\begin{alltt}
[1] "4" "5" "6" "7" "8" "9"
\end{alltt}
\normalsize
\mytem letters[3:8], LETTERS[3:8]
\mytem data.frame(letters[3:8], letters[8:3], {\color{red!13!blue}{stringsAsFactors = F}})
\end{itemize}
\end{frame}
\section{чтение}
\begin{frame}[fragile]{Чтение файлов?}
\begin{itemize}
\mytem z <- read.table("a.csv", sep = "\dots") \hfill \# текст и разделители
\mytem y <- read.csv("b.txt") \hfill \# разделитель "{},{}"
\mytem x <- read.csv2("c.csv") \hfill \# разделитель ";"
\mytem w <- read.delim("d.txt") \hfill \# разделитель "\textbackslash t"
\mytem v <- readLines("e.txt") \hfill \# чтение "голого"{} текста
\end{itemize}
\end{frame}
\section{манипуляции}
\subsection{соединение}
\begin{frame}[fragile]{Соединение строк: paste()}
\begin{itemize}
\mytem paste("Школа"{},  "Лингвистики")
\footnotesize
\begin{alltt}
[1] "Школа Лингвистики"
\end{alltt}
\normalsize
\vfill
\mytem paste(9, 7, 8, {\color{red!13!blue}{sep = "{}-{}"{}}}) \hfill \# другой разделитель
\footnotesize
\begin{alltt}
[1] "9-7-8"
\end{alltt}
\normalsize
\vfill
\mytem paste({\color{red!13!blue}{LETTERS[1:5]}}, {\color{red!27!green}{letters[1:5]}}, sep = "{}"{}) \hfill \# векторы
\footnotesize
\begin{alltt}
[1] "{\color{red!13!blue}{A}}{\color{red!27!green}{a}}" "{\color{red!13!blue}{B}}{\color{red!27!green}{b}}" "{\color{red!13!blue}{C}}{\color{red!27!green}{c}}" "{\color{red!13!blue}{D}}{\color{red!27!green}{d}}" "{\color{red!13!blue}{E}}{\color{red!27!green}{e}}"
\end{alltt}
\normalsize
\vfill
\mytem paste({\color{red!13!blue}{"F"{}}}, {\color{red!27!green}{letters[1-5]}}, sep = "{}"{}) \hfill \# векторы разной длины
\footnotesize
\begin{alltt}
[1] "{\color{red!13!blue}{F}}{\color{red!27!green}{a}}{}" "{\color{red!13!blue}{F}}{\color{red!27!green}{b}}{}" "{\color{red!13!blue}{F}}{\color{red!27!green}{c}}{}" "{\color{red!13!blue}{F}}{\color{red!27!green}{d}}{}" "{\color{red!13!blue}{F}}{\color{red!27!green}{e}}{}"
\end{alltt}
\normalsize
\vfill
\mytem paste({\color{red!13!blue}{"F"{}}}, letters[1-5], sep = "{}"{}, collapse = "{ }"{}) \hfill \# в один вектор
\footnotesize
\begin{alltt}
[1] "{\color{red!13!blue}{F}}{\color{red!27!green}{a}} {\color{red!13!blue}{F}}{\color{red!27!green}{b}} {\color{red!13!blue}{F}}{\color{red!27!green}{c}} {\color{red!13!blue}{F}}{\color{red!27!green}{d}} {\color{red!13!blue}{F}}{\color{red!27!green}{e}}{}"
\end{alltt}
\normalsize
\vfill
\mytem paste({\color{red!13!blue}{"F"{}}}, {\color{red!27!green}{letters[1-5]}}, sep = "{}"{}, collapse = "{}-{}"{}) \hfill \# другой разделитель
\footnotesize
\begin{alltt}
[1] "{\color{red!13!blue}{F}}{\color{red!27!green}{a}}-{\color{red!13!blue}{F}}{\color{red!27!green}{b}}-{\color{red!13!blue}{F}}{\color{red!27!green}{c}}-{\color{red!13!blue}{F}}{\color{red!27!green}{d}}-{\color{red!13!blue}{F}}{\color{red!27!green}{e}}{}"
\end{alltt}
\normalsize
\vfill
\mytem paste(c({\color{red!13!blue}{"F"{}}}, {\color{red!27!green}{letters[1-5]}}), collapse = "{}-{}"{}) \hfill \# развекторизовать
\footnotesize
\begin{alltt}
[1] "{\color{red!13!blue}{F}}-{\color{red!27!green}{a}}-{\color{red!27!green}{b}}-{\color{red!27!green}{c}}-{\color{red!27!green}{d}}-{\color{red!27!green}{e}}{}"
\end{alltt}
\normalsize
\end{itemize}
\end{frame}
\begin{frame}[fragile]{Соединение строк: library(stringr)}
\begin{itemize}
\mytem str\_c("Школа"{},  "Лингвистики")
\footnotesize
\begin{alltt}
[1] "ШколаЛингвистики"
\end{alltt}
\normalsize
\vfill
\mytem str\_c("Школа"{},  "Лингвистики"{}, sep = "{ }"{}) \hfill \# разделитель
\footnotesize
\begin{alltt}
[1] "Школа Лингвистики"
\end{alltt}
\normalsize
\mytem str\_c({\color{red!13!blue}{LETTERS[1:5]}}, {\color{red!27!green}{letters[1:5]}}, sep = "{}"{}) \hfill \# векторы
\footnotesize
\begin{alltt}
[1] "{\color{red!13!blue}{A}}{\color{red!27!green}{a}}" "{\color{red!13!blue}{B}}{\color{red!27!green}{b}}" "{\color{red!13!blue}{C}}{\color{red!27!green}{c}}" "{\color{red!13!blue}{D}}{\color{red!27!green}{d}}" "{\color{red!13!blue}{E}}{\color{red!27!green}{e}}"
\end{alltt}
\normalsize
\vfill
\mytem str\_c({\color{red!13!blue}{"F"{}}}, {\color{red!27!green}{letters[1-5]}}, sep = "{}"{}) \hfill \# векторы разной длины
\footnotesize
\begin{alltt}
[1] "{\color{red!13!blue}{F}}{\color{red!27!green}{a}}{}" "{\color{red!13!blue}{F}}{\color{red!27!green}{b}}{}" "{\color{red!13!blue}{F}}{\color{red!27!green}{c}}{}" "{\color{red!13!blue}{F}}{\color{red!27!green}{d}}{}" "{\color{red!13!blue}{F}}{\color{red!27!green}{e}}{}"
\end{alltt}
\normalsize
\vfill
\mytem str\_c({\color{red!13!blue}{"F"{}}}, letters[1-5], sep = "{}"{}, collapse = "{ }"{}) \hfill \# в один вектор
\footnotesize
\begin{alltt}
[1] "{\color{red!13!blue}{F}}{\color{red!27!green}{a}} {\color{red!13!blue}{F}}{\color{red!27!green}{b}} {\color{red!13!blue}{F}}{\color{red!27!green}{c}} {\color{red!13!blue}{F}}{\color{red!27!green}{d}} {\color{red!13!blue}{F}}{\color{red!27!green}{e}}{}"
\end{alltt}
\normalsize
\vfill
\mytem str\_c({\color{red!13!blue}{"F"{}}}, {\color{red!27!green}{letters[1-5]}}, sep = "{}"{}, collapse = "{}-{}"{}) \hfill \# другой разделитель
\footnotesize
\begin{alltt}
[1] "{\color{red!13!blue}{F}}{\color{red!27!green}{a}}-{\color{red!13!blue}{F}}{\color{red!27!green}{b}}-{\color{red!13!blue}{F}}{\color{red!27!green}{c}}-{\color{red!13!blue}{F}}{\color{red!27!green}{d}}-{\color{red!13!blue}{F}}{\color{red!27!green}{e}}{}"
\end{alltt}
\normalsize
\vfill
\mytem str\_c(c({\color{red!13!blue}{"F"{}}}, {\color{red!27!green}{letters[1-5]}}), collapse = "{}-{}"{}) \hfill \# развекторизовать
\footnotesize
\begin{alltt}
[1] "{\color{red!13!blue}{F}}-{\color{red!27!green}{a}}-{\color{red!27!green}{b}}-{\color{red!27!green}{c}}-{\color{red!27!green}{d}}-{\color{red!27!green}{e}}{}"
\end{alltt}
\normalsize
\end{itemize}
\end{frame}
\begin{frame}[fragile]{Соединение строк: paste() vs. str\_c()}
\begin{itemize}
\item[paste] нативная функция
\item[paste] paste("1"{}, "3")
\footnotesize
\begin{alltt}
[1] "1 3"
\end{alltt}
\normalsize
\item[paste] paste("f"{}, NULL, "{}"{}, "g"{}, sep = "{}-{}")
\footnotesize
\begin{alltt}
[1] "f---g"
\end{alltt}
\normalsize
\vfill
\item[str\_c] library(stringr)
\item[str\_c] str\_c("1"{}, "3")
\footnotesize
\begin{alltt}
[1] "13"
\end{alltt}
\normalsize
\item[str\_c] str\_c("f"{}, NULL, "{}"{}, "g"{}, sep = "{}-{}")
\footnotesize
\begin{alltt}
[1] "f--g"
\end{alltt}
\normalsize
\end{itemize}
\end{frame}
\subsection{количество символов}
\begin{frame}{Количество символов}
a <- "the quick brown fox jumps over the lazy dog"
\begin{itemize}
\mytem nchar(a)
\footnotesize
\begin{alltt}
[1] 43
\end{alltt}
\normalsize
\mytem library(stringr); str\_length(a)
\footnotesize
\begin{alltt}
[1] 43
\end{alltt}
\normalsize
\end{itemize}
\vfill
b  <-  c("the"{}, "quick"{}, "brown"{}, "fox")
\begin{itemize}
\mytem nchar(b)
\footnotesize
\begin{alltt}
[1] 3 5 5 3
\end{alltt}
\normalsize
\mytem str\_length(b)
\footnotesize
\begin{alltt}
[1] 3 5 5 3
\end{alltt}
\normalsize
\end{itemize}
\vfill
c <- c("the"{}, "quick"{}, NULL, "brown"{}, "{}"{}, "fox")
\begin{itemize}
\mytem nchar(c)
\footnotesize
\begin{alltt}
[1] 3 5 5 0 3
\end{alltt}
\normalsize
\mytem str\_length(c)
\footnotesize
\begin{alltt}
[1] 3 5 5 0 3
\end{alltt}
\normalsize
\end{itemize}
\end{frame}
\subsection{регистр букв}
\begin{frame}{Изменение регистра}
a <- "the QuiCk bRown fOx juMps ovEr tHe laZy dOg"
\begin{itemize}
\mytem tolower(a)
\footnotesize
\begin{alltt}
[1] "the quick brown fox jumps over the lazy dog"
\end{alltt}
\normalsize
\mytem toupper(a)
\footnotesize
\begin{alltt}
[1] "THE QUICK BROWN FOX JUMPS OVER THE LAZY DOG"
\end{alltt}
\normalsize
\end{itemize}
\vfill
b <- "любЯ, сЪешЬ щипЦы, — взДохнёт мэр, — кайф Жгуч{}"
\begin{itemize}
\mytem tolower(b)
\footnotesize
\begin{alltt}
[1] "любя, съешь щипцы, — вздохнёт мэр, — кайф жгуч"
\end{alltt}
\normalsize
\mytem toupper(b)
\footnotesize
\begin{alltt}
[1] "ЛЮБЯ, СЪЕШЬ ЩИПЦЫ, — ВЗДОХНЁТ МЭР, — КАЙФ ЖГУЧ"
\end{alltt}
\normalsize
\end{itemize}
\end{frame}
\subsection{подстроки}
\begin{frame}{Выделение подстроки: substr()}
a <- "the quick brown fox"\\
b <- c("the"{}, "quick"{}, "brown"{}, "fox"{})
\begin{itemize}
\mytem substr(a, 2, 3) \hfill \# второй и третий символ
\footnotesize
\begin{alltt}
[1] "he"
\end{alltt}
\normalsize
\vfill
\mytem substr(b, 2, 3) \hfill \# второй и третий символ
\footnotesize
\begin{alltt}
[1] "he"{ }"ui"{ }"ro"{ }"ox"
\end{alltt}
\normalsize
\vfill
\mytem substr(b, 2, 2) <- "."; b \hfill \# заменяет второй символ точкой
\footnotesize
\begin{alltt}
[1] "t.e"{ }"q.ick"{ }"b.own"{ }"f.x"
\end{alltt}
\normalsize
\vfill
\mytem substr(b, 2, 2) <- c("1"{}, "2") ; b \hfill \# заменяет второй символ вектором
\footnotesize
\begin{alltt}
[1] "t1e"{ }"q2ick"{ }"b1own"{ }"f2x"
\end{alltt}
\normalsize
\end{itemize}
\end{frame}
\begin{frame}{Выделение подстроки: substring()}
a <- "the quick brown fox"\\
b <- c("the"{}, "quick"{}, "brown"{}, "fox"{})
\begin{itemize}
\mytem substring(b, 2, 3) \hfill \# второй и третий символ
\footnotesize
\begin{alltt}
[1] "he"{ }"ui"{ }"ro"{ }"ox"
\end{alltt}
\normalsize
\vfill
\mytem substring(b, 2, 2) <- "."; b \hfill \# заменяет второй символ точкой
\footnotesize
\begin{alltt}
[1] "t.e"{ }"q.ick"{ }"b.own"{ }"f.x"
\end{alltt}
\normalsize
\vfill
\mytem substring(b, 2, 2) <- c("1"{}, "2") ; b \hfill \# заменяет второй cимвол вектором
\footnotesize
\begin{alltt}
[1] "t1e"{ }"q2ick"{ }"b1own"{ }"f2x"
\end{alltt}
\normalsize
\vfill
\mytem substring(b, 2) <- c("1"{}, "22"{}, "333") ; b \hfill \# заменяет второй cимвол вектором
\footnotesize
\begin{alltt}
[1] "t1e"{ }"q22ck"{ }"b333n"{ }"f1x"
\end{alltt}
\normalsize
\vfill
\mytem substring(a, 1:19, 1:19) \hfill \# извлекает каждую букву
\scriptsize
\begin{alltt}
[1] "t"{ }"h"{ }"e"{ }"{ }"{ }"q"{ }"u"{ }"i"{ }"c"{ }"k"{ }"{ }"{ }"b"{ }"r"{ }"o"{ }"w"{ }"n"{ }"{ }"{ }"f"{ }"o"{ }"x"
\end{alltt}
\normalsize
\end{itemize}
\end{frame}
\begin{frame}{Выделение подcтроки: library(stringr)}
a <- "the quick brown fox"\\
b <- c("the"{}, "quick"{}, "brown"{}, "fox"{})
\begin{itemize}
\mytem str\_sub(b, 2, 3) \hfill \# второй и третий cимвол
\footnotesize
\begin{alltt}
[1] "he"{ }"ui"{ }"ro"{ }"ox"
\end{alltt}
\normalsize
\vfill
\mytem str\_sub(b, 2, 2) <- "."; b \hfill \# заменяет второй cимвол точкой
\footnotesize
\begin{alltt}
[1] "t.e"{ }"q.ick"{ }"b.own"{ }"f.x"
\end{alltt}
\normalsize
\vfill
\mytem str\_sub(b, 2, 2) <- ".;."; b \hfill \# заменяет второй cимвол другими
\footnotesize
\begin{alltt}
[1] "t.;.e"{ }"q.;.ick"{ }"b.;.own"{ }"f.;.x"
\end{alltt}
\normalsize
\vfill
\mytem str\_sub(b, 2) <- c("1"{}, "2") ; b \hfill \# заменяет второй cимвол вектором
\footnotesize
\begin{alltt}
[1] "t1e"{ }"q2ick"{ }"b1own"{ }"f2x"
\end{alltt}
\normalsize
\vfill
\mytem str\_sub(a, 1:19, 1:19) \hfill \# извлекает каждую букву
\scriptsize
\begin{alltt}
[1] "t"{ }"h"{ }"e"{ }"{ }"{ }"q"{ }"u"{ }"i"{ }"c"{ }"k"{ }"{ }"{ }"b"{ }"r"{ }"o"{ }"w"{ }"n"{ }"{ }"{ }"f"{ }"o"{ }"x"
\end{alltt}
\normalsize
\vfill
\mytem str\_sub(b, -1, -1) \hfill \# поcледний символ
\footnotesize
\begin{alltt}
[1] "e"{ }"k"{ }"n"{ }"x"
\end{alltt}
\normalsize
\end{itemize}
\end{frame}
\begin{frame}[fragile]{Выделение подстроки: library(stringr)}
a <- c("the quick brown fox jumps over"{})\\
b <- c("jumps over the lazy dog"{})
\begin{itemize}
\vfill
\mytem word(a, 2) \hfill \# второе слово
\footnotesize
\begin{alltt}
[1] "quick"
\end{alltt}
\normalsize
\vfill
\mytem word(c(b, a), 4) \hfill \# четвертое слово каждой строки
\footnotesize
\begin{alltt}
[1] "lazy"{ }"fox"
\end{alltt}
\normalsize
\vfill
\mytem word(c(b, a), -3) \hfill \# третье слово с конца каждой строки
\footnotesize
\begin{alltt}
[1] "the"{ }"fox"
\end{alltt}
\normalsize
\vfill
\mytem word(c(b, a), 2, 4) \hfill \# со второго по четвертое слово каждой строки
\footnotesize
\begin{alltt}
[1] "over the lazy"{ }"quick brown fox"
\end{alltt}
\normalsize
\vfill
\mytem word(c(b, a), -2, -1) \hfill \# последние два слова каждой строки
\footnotesize
\begin{alltt}
[1] "lazy dog"{ }"jumps over"
\end{alltt}
\normalsize
\end{itemize}
\end{frame}
\subsection{поиск}
\begin{frame}[fragile]{Поиск вектора по фрагменту: grep()}
a <- c("the quick"{}, "brown"{}, "fox"{}, "jumps"{}, "over"{})\\
b <- c("jumps"{}, "over"{}, "the"{}, "lazy"{}, "dog"{})
\begin{itemize}
\vfill
\mytem grep("the"{}, c(a, b)) \hfill номер элемента, содержащего подстроку
\footnotesize
\begin{alltt}
[1] 1 8
\end{alltt}
\normalsize
\vfill
\mytem grep("the"{}, c(a, b), value = T) \hfill элемент, содержащий подстроку
\footnotesize
\begin{alltt}
[1] "the quick" "the"
\end{alltt}
\normalsize
\vfill
\mytem grep("the"{}, c(a, b), invert = T) \hfill номер элемента, не содержащего подстроку
\footnotesize
\begin{alltt}
[1]  2  3  4  5  6  7  9 10
\end{alltt}
\normalsize
\vfill
\mytem grep("the"{}, c(a, b), invert = T, value = T) \hfill элемент, не содержащий подстроку
\footnotesize
\begin{alltt}
[1] "brown" "fox"   "jumps" "over"  "jumps" "over"  "lazy"  "dog"
\end{alltt}
\normalsize
\vfill
\mytem grepl("the"{}, c(a, b)) \hfill элемент, содержащий подстроку
\scriptsize
\begin{alltt}
[1] TRUE FALSE FALSE FALSE FALSE FALSE FALSE  TRUE FALSE FALSE
\end{alltt}
\normalsize
\end{itemize}
\end{frame}
\begin{frame}[fragile]{Поиск вектора по фрагменту: library("stringr")}
a <- c("the quick"{}, "brown"{}, "fox"{}, "jumps"{}, "over"{})\\
b <- c("jumps"{}, "over"{}, "the"{}, "lazy"{}, "dog"{})
\begin{itemize}
\mytem str\_detect("the"{}, c(a, b)) \hfill элемент, содержащий подстроку
\scriptsize
\begin{alltt}
[1] TRUE FALSE FALSE FALSE FALSE FALSE FALSE  TRUE FALSE FALSE
\end{alltt}
\normalsize
\mytem c(a,b)[str\_detect("the", c(a, b))] \hfill элемент, содержащий подстроку
\scriptsize
\begin{alltt}
[1] "the" "the"
\end{alltt}
\normalsize
\end{itemize}
\end{frame}
\subsection{замена}
\begin{frame}[fragile]{Замена фрагмента}
a <- c("the quick brown fox jumps over"{})\\
b <- c("jumps over the lazy dog"{})
\begin{itemize}
\vfill
\mytem sub("o"{}, "\_"{}, c(a, b)) \hfill \# одно вхождение
\footnotesize
\begin{alltt}
[1] "the quick br_wn fox jumps over" "jumps _ver the lazy dog"  
\end{alltt}
\normalsize
\vfill
\mytem gsub("o"{}, "\_"{}, c(a, b)) \hfill \# все вхождения
\footnotesize
\begin{alltt}
[1] "the quick br_wn f_x jumps _ver" "jumps _ver the lazy d_g"
\end{alltt}
\normalsize
\vfill
\mytem library("stringr"); str\_replace("o"{}, "\_"{}, c(a, b)) \hfill \# одно вхождение
\footnotesize
\begin{alltt}
[1] "the quick br_wn fox jumps over" "jumps _ver the lazy dog" 
\end{alltt}
\normalsize
\vfill
\mytem library("stringr"); str\_replace\_all("o"{}, "\_"{}, c(a, b)) \hfill \# все вхождения
\footnotesize
\begin{alltt}
[1] "the quick br\_wn f\_x jumps \_ver" "jumps \_ver the lazy d\_g"
\end{alltt}
\normalsize
\end{itemize}
\end{frame}
\subsection{разбиение}
\begin{frame}[fragile]{Разбиение}
a <- c("the quick brown fox jumps over"{})\\
b <- c("jumps over the lazy dog"{})
\begin{itemize}
\vfill
\mytem strsplit(c(a,b),{ }"{ }"{}) \hfill \# возвращает список слов
\mytem library("stringr"); str\_split(c(a,b),{ }"{ }"{}) \hfill \# возвращает список слов
\end{itemize}
\end{frame}
\section{векторы строк}
\subsection{объединение}
\begin{frame}[fragile]{Вектор уникальных строк}
a <- c("the"{}, {\color{red!13!blue}{"quick"{}, "brown"{}, "fox"{}}}, "jumps"{}, "over"{})\\
b <- c("jumps"{}, "over"{}, "the"{}, {\color{red!27!green}{"lazy"{}, "dog"{}}})
\begin{itemize}
\mytem union(a, b)
\footnotesize
\begin{alltt}
[1] "the"{ }{\color{red!13!blue}{"quick"{ }"brown"{ }"fox"{}}}{ }"jumps"{ }"over"{ }{\color{red!27!green}{"lazy"{ }"dog"{}}}  
\end{alltt}
\normalsize
\end{itemize}
\end{frame}
\subsection{пересечение}
\begin{frame}[fragile]{Пересечение векторов строк}
a <- c("the"{}, {\color{red!13!blue}{"quick"{}, "brown"{}, "fox"{}}}, "jumps"{}, "over"{})\\
b <- c("jumps"{}, "over"{}, "the"{}, {\color{red!27!green}{"lazy"{}, "dog"{}}})
\begin{itemize}
\mytem intersect(a, b)
\footnotesize
\begin{alltt}
[1] "the"{ }"jumps"{ }"over"
\end{alltt}
\normalsize
\end{itemize}
\end{frame}
\subsection{разность}
\begin{frame}[fragile]{Разность векторов строк}
a <- c("the"{}, {\color{red!13!blue}{"quick"{}, "brown"{}, "fox"{}}}, "jumps"{}, "over"{})\\
b <- c("jumps"{}, "over"{}, "the"{}, {\color{red!27!green}{"lazy"{}, "dog"{}}})
\begin{itemize}
\mytem setdiff(a, b)
\footnotesize
\begin{alltt}
[1] {\color{red!13!blue}{"quick"{ }"brown"{ }"fox"}}
\end{alltt}
\normalsize
\end{itemize}
\end{frame}
\subsection{сравнение}
\begin{frame}[fragile]{Сравнение векторов строк}
a <- c("the"{}, {\color{red!13!blue}{"quick"{}, "brown"{}, "fox"{}}}, "jumps"{}, "over"{})\\
b <- c("jumps"{}, "over"{}, "the"{}, {\color{red!27!green}{"lazy"{}, "dog"{}}})
\begin{itemize}
\mytem setequal(a, b)
\footnotesize
\begin{alltt}
[1] FALSE
\end{alltt}
\normalsize
\end{itemize}
\vfill
a <- c("the"{}, "quick"{}, "brown"{}, "fox"{}, "jumps"{}, "over"{})\\
c <- c("brown"{}, "fox"{}, "jumps"{}, "over"{}, "the"{}, "quick"{})
\begin{itemize}
\mytem setequal(c, a)
\footnotesize
\begin{alltt}
[1] TRUE
\end{alltt}
\normalsize
\end{itemize}
\vfill
a <- c("the"{}, "quick"{}, "brown"{}, "fox"{}, "jumps"{}, "over"{})\\
c <- c("brown"{}, "fox"{}, "jumps"{}, "over"{}, "the"{}, "quick"{})
\begin{itemize}
\mytem identical(c, a)
\footnotesize
\begin{alltt}
[1] FALSE
\end{alltt}
\normalsize
\end{itemize}
\end{frame}
\subsection{подмножество}
\begin{frame}[fragile]{Является ли вектор подмножеством другого?}
a <- c("the"{}, {\color{red!13!blue}{"quick"{}, "brown"{}, "fox"{}}})\\
b <- "the"\\
c <- "lazy"\\
d <- "dog"
\begin{itemize}
\vfill
\mytem is.element(b, a)
\footnotesize
\begin{alltt}
[1] TRUE
\end{alltt}
\normalsize
\vfill
\mytem is.element(c, a)
\footnotesize
\begin{alltt}
[1] FALSE
\end{alltt}
\normalsize
\vfill
\mytem is.element(c(b, c, d), a)
\footnotesize
\begin{alltt}
[1] TRUE FALSE FALSE
\end{alltt}
\normalsize
\end{itemize}
\end{frame}
\section{регулярки}
\subsection{метасимволы}
\begin{frame}[fragile]{Экранирование метасимволов}
В R для экранирования метасимволов используется двойной слэш (а для экранирования слэша аж три слэша):\\
\footnotesize
\begin{alltt}
\textbackslash\textbackslash{\color{red!13!blue}{.}}
\textbackslash\textbackslash{\color{red!13!blue}{\$}}
\textbackslash\textbackslash{\color{red!13!blue}{*}}
\textbackslash\textbackslash{\color{red!13!blue}{+}}
\textbackslash\textbackslash{\color{red!13!blue}{?}}
\textbackslash\textbackslash{\color{red!13!blue}{|}}
\textbackslash\textbackslash\textbackslash{\color{red!13!blue}{\textbackslash}}
\textbackslash\textbackslash{\color{red!13!blue}{^}}
\textbackslash\textbackslash{\color{red!13!blue}{[}}
\textbackslash\textbackslash{\color{red!13!blue}{]}}
\textbackslash\textbackslash{\color{red!13!blue}{\{}}
\textbackslash\textbackslash{\color{red!13!blue}{\}}}
\textbackslash\textbackslash{\color{red!13!blue}{(}}
\textbackslash\textbackslash{\color{red!13!blue}{)}}
\end{alltt}
\normalsize
\end{frame}
\begin{frame}[fragile]{Классы знаков}
\footnotesize
\begin{alltt}
\textbackslash\textbackslash{\color{red!13!blue}{d}} \hfill \# цифры
\textbackslash\textbackslash{\color{red!13!blue}{D}} \hfill \# не цифры
\textbackslash\textbackslash{\color{red!13!blue}{s}} \hfill \# пробел
\textbackslash\textbackslash{\color{red!13!blue}{S}} \hfill \# не пробел
\textbackslash\textbackslash{\color{red!13!blue}{w}} \hfill \# слово
\textbackslash\textbackslash{\color{red!13!blue}{W}} \hfill \# не слово
\end{alltt}
\normalsize
\vfill
\begin{multicols}{2}
\begin{itemize} 
\mytem gsub("\textbackslash\textbackslash d"{}, "\textbackslash\textbackslash.{}"{}, "2016 год")
\footnotesize
\begin{alltt}
[1] ".... год"
\end{alltt}
\normalsize
\mytem gsub("\textbackslash\textbackslash D"{}, "\textbackslash\textbackslash.{}"{}, "2016 год")
\footnotesize
\begin{alltt}
[1] "2016...."
\end{alltt}
\normalsize
\mytem gsub("\textbackslash\textbackslash s"{}, "\textbackslash\textbackslash.{}"{}, "2016 год")
\footnotesize
\begin{alltt}
[1] "2016.год"
\end{alltt}
\normalsize
\columnbreak
\mytem gsub("\textbackslash\textbackslash S"{}, "\textbackslash\textbackslash.{}"{}, "2016 год")
\footnotesize
\begin{alltt}
[1] ".... ..."
\end{alltt}
\normalsize
\mytem gsub("\textbackslash\textbackslash w"{}, "\textbackslash\textbackslash.{}"{}, "2016 год")
\footnotesize
\begin{alltt}
[1] ".... ..."
\end{alltt}
\normalsize
\mytem gsub("\textbackslash\textbackslash W"{}, "\textbackslash\textbackslash.{}"{}, "2016 год")
\footnotesize
\begin{alltt}
[1] "2016.год"
\end{alltt}
\normalsize
\end{itemize}
\end{multicols}
\end{frame}
\subsection{наборы символов}
\begin{frame}[fragile]{Наборы символов}
\begin{multicols}{2}
\begin{itemize} 
\mytem gsub("[0-9]"{}, "\textbackslash\textbackslash.{}"{}, "2016 год")
\footnotesize
\begin{alltt}
[1] ".... год"
\end{alltt}
\normalsize
\vfill
\mytem gsub("[a-z]"{}, "\textbackslash\textbackslash.{}"{}, "2016 year")
\footnotesize
\begin{alltt}
[1] "2016 ....{}"
\end{alltt}
\normalsize
\vfill
\mytem gsub("[а-я]"{}, "\textbackslash\textbackslash.{}"{}, "2016 год")
\footnotesize
\begin{alltt}
[1] "2016 ...{}"
\end{alltt}
\normalsize
\columnbreak
\mytem gsub("[A-Z]"{}, "\textbackslash\textbackslash.{}"{}, "2016 yEar")
\footnotesize
\begin{alltt}
[1] "2016 y.ar"
\end{alltt}
\normalsize
\vfill
\mytem gsub("[А-Я]"{}, "\textbackslash\textbackslash.{}"{}, "2016 гОД")
\footnotesize
\begin{alltt}
[1] "2016 г..{}"
\end{alltt}
\normalsize
\vfill
\mytem gsub("[a-zA-Z]"{}, "\textbackslash\textbackslash.{}"{}, "yEar")
\footnotesize
\begin{alltt}
[1] "...{}."
\end{alltt}
\normalsize
\vfill
\mytem gsub("[\^{}а-я]"{}, "\textbackslash\textbackslash.{}"{}, "2016 год")
\footnotesize
\begin{alltt}
[1] "....год"
\end{alltt}
\normalsize
\end{itemize}
\end{multicols}
\end{frame}
\subsection{квантификация}
\begin{frame}[fragile]{Квантификаторы}
a <- c("тара"{}, "коса"{}, "шоссе"{}, "касса"{}, "масса")
\begin{itemize}
\mytem gsub("сс{\color{red!13!blue}{?}}"{}, "\textbackslash\textbackslash."{}, a) \hfill \# ноль или один раз
\scriptsize
\begin{alltt}
[1] "тара" "ко.а" "шо.е" "ка.а" "ма.а"
\end{alltt}
\normalsize
\mytem gsub("сс{\color{red!13!blue}{*}}"{}, "\textbackslash\textbackslash."{}, a) \hfill \# ноль и более раз
\scriptsize
\begin{alltt}
[1] "тара" "ко.а" "шо.е" "ка.а" "ма.а"
\end{alltt}
\normalsize
\mytem gsub("сс{\color{red!13!blue}{+}}"{}, "\textbackslash\textbackslash."{}, a) \hfill \# один и более раз
\scriptsize
\begin{alltt}
[1] "тара" "коса" "шо.е" "ка.а" "ма.а"
\end{alltt}
\normalsize
\mytem gsub("а{\color{red!13!blue}{.}}а"{}, "\textbackslash\textbackslash."{}, a) \hfill \# любой символ
\scriptsize
\begin{alltt}
[1] "т."    "коса"  "шоссе" "касса" "масса"
\end{alltt}
\normalsize
\mytem gsub("а{\color{red!13!blue}{.*}}а"{}, "\textbackslash\textbackslash."{}, a) \hfill \# любой символ ноль и более раз
\scriptsize
\begin{alltt}
[1] "т."    "коса"  "шоссе" "к." "м."
\end{alltt}
\normalsize
\mytem gsub("с{\color{red!13!blue}{\{2\}}}"{}, "\textbackslash\textbackslash."{}, a) \hfill \# два раза
\scriptsize
\begin{alltt}
[1] "тара"    "коса"  "шоссе" "ка.а" "ма.а"
\end{alltt}
\normalsize
\mytem gsub("с{\color{red!13!blue}{\{1,\}}}"{}, "\textbackslash\textbackslash."{}, a) \hfill \# один и более раз
\scriptsize
\begin{alltt}
[1] "тара"    "ко.а"  "шоссе" "ка.а" "ма.а"
\end{alltt}
\normalsize
\mytem gsub("с{\color{red!13!blue}{\{1,2\}}}"{}, "\textbackslash\textbackslash."{}, a) \hfill \# от одного до двух раз
\scriptsize
\begin{alltt}
[1] "тара"    "ко.а"  "шоссе" "ка.а" "ма.а"
\end{alltt}
\normalsize
\end{itemize}
\vspace{-2mm}
{\color{red!13!blue}{Пробел}}: \scriptsize\verb*"с{1,2}"\normalsize{} — правильно, \scriptsize\verb*"с{1, 2}"\normalsize{} — неправильно.
\end{frame}
\section{}
\begin{frame}
{\huge Спасибо за внимание!\bigskip\\
\normalsize Пишите письма\\
agricolamz@gmail.com
\vspace{-130pt}}
\end{frame}
\end{document}